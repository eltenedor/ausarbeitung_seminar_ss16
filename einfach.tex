\section{Einfache Wurzeln}
\label{sec:einfach}

In diesem Abschnitt sollen einige Eigenschaften einfacher Wurzeln bewiesen werden.  
Im Folgenden bezeichne $\Delta$ ein fest gewähltes Fundamentalsystem des Wurzelsystems $\Phi$ über dem \euklid ischen Vektorraum $E$.

\begin{lem}
  \label{lem:independentSet}
  Sei $M \subseteq E$ eine Teilmenge von Vektoren und $P_\gamma$ eine Hyperebene.
  Es gelte $(\alpha, \gamma) > 0$ für alle $\alpha \in M$ und zudem $(\alpha, \beta) \leq 0$ für alle paarweise verschiedenen $\alpha, \beta \in M$.
  Dann ist $M$ linear unabhängig.
\end{lem}

\begin{proof}
  Wir betrachten die Gleichung $\sum_{\alpha \in M} k_\alpha \alpha = 0$ und definieren die disjunkten Mengen $A := \{\alpha \in M \mid k_\alpha < 0\}$ und  $B := \{\beta \in M \mid k_\beta > 0\}$. 
  Damit gilt 
  \begin{displaymath}
    \sum_{\alpha \in A} k_\alpha \alpha = \sum_{\beta \in B} k_\beta \beta \tag{$\ast$}.
  \end{displaymath}
  Daraus folgt mit $\varepsilon := \sum_{\alpha \in A} k_\alpha \alpha$ dann
  \begin{displaymath}
    0 
    \leq (\varepsilon, \varepsilon) 
    = (\;\sum_{\alpha \in A} k_\alpha \alpha,\sum_{\beta \in B} k_\beta \beta\;) 
    = \sum_{\alpha \in A, \beta \in B} k_\alpha k_\beta \; (\alpha, \beta) 
    \leq 0,
  \end{displaymath}
  aufgrund der Eigenschaften des Skalarprodukts und der Voraussetzung $(\alpha, \beta) \leq 0$ für alle paarweise verschiedenen $\alpha, \beta$.
  Dies impliziert $\varepsilon = 0$, womit auch 
  \begin{displaymath}
    0 = (\varepsilon, \gamma) = \sum_{\alpha \in A} k_\alpha (\alpha, \gamma)
  \end{displaymath}
  folgt. Da voraussetzungsgemäß $(\alpha, \gamma) > 0$ gilt, müssen alle $k_\alpha$ mit $\alpha \in A$ bereits $0$ sein.
  Analog folgert man unter Verwendung der Identität ($\ast$), dass alle $k_\beta$ mit $\beta \in B$ gleich $0$ sein müssen.
\end{proof}

Das folgende Lemma demonstriert, wie sich aus einer positiven Wurzel neue positive Wurzeln gewinnen lassen.

\begin{lem}
  \label{lem:posRootNewRoot}
  Ist $\alpha \in \Phi$ eine positive aber nicht einfache Wurzel, so existiert ein $\beta \in \Delta$, sodass die Differenz $\alpha - \beta$ eine positive Wurzel ist.
\end{lem}

\begin{proof}
  Wir wollen zeigen, dass ein $\beta \in \Delta$ existiert mit $(\alpha, \beta) > 0$.
  Angenommen, für alle $\beta \in \Delta$ gelte $(\alpha, \beta) \leq 0$.
  Zudem gilt für alle $\beta, \beta' \in \Delta$ immer $(\beta, \beta') \leq 0$, denn sonst wäre nach Lemma \ref{lem:sumDiffRoot} auch $\beta - \beta'$ eine Wurzel, was jedoch \hyperref[it:B2]{(B2)} widerspricht.
  Damit sind die Voraussetzungen für Lemma \ref{lem:independentSet} erfüllt und es ist $\Delta \cup \{\alpha\}$ eine linear unabhängige Menge.
  Dies widerspricht jedoch der Voraussetzung, dass $\Delta$ ein Fundamentalsystem ist.

  Also muss ein $\beta \in \Delta$ existieren mit $(\alpha, \beta) > 0$.
  Zudem kann $\beta$ nicht proportional zu $\alpha$ sein, da sonst $\alpha$ nicht positiv sein könnte.
  Unter Verwendung von Lemma \ref{lem:sumDiffRoot} folgt somit, dass $\alpha - \beta$ eine Wurzel ist.
  
  Bezüglich $\Delta$ lässt sich $\alpha$ auch schreiben als $\alpha = \sum_{\gamma \in \Delta} k_\gamma \gamma$, wobei aufgrund der Positivität von $\alpha$ alle $k_\gamma \geq 0$ sind und darüber hinaus auch mindestens ein $\gamma \neq \beta$ mit  $k_\gamma > 0$ existiert.
  Ansonsten wäre nämlich $\alpha$ proportional zu $\beta$. 
  Damit würde jedoch nach \hyperref[it:R2]{(R2)} entweder $\alpha = \beta$ gelten, was der Voraussetzung, dass $\alpha$ nicht einfach ist, widerspricht, oder es würde $\alpha = -\beta$ gelten, was der vorausgesetzten Positivität von $\alpha$ widerspricht.

  Betrachtet man nun die Darstellung $\alpha - \beta = \sum_{\gamma \in \Delta \setminus \{\beta\}} k_\gamma \gamma + (- \beta)$, so besitzt diese Summe zumindest einen strikt positiven Linearfaktor.
  Die Eigenschaft \hyperref[it:B2]{(B2)} erzwingt nun die Positivität aller restlichen Linearfaktoren und folglich ist auch $\alpha - \beta$ eine positive Wurzel.
\end{proof}
 
\begin{cor}
  Jedes $\beta \in \Phi^+$ lässt sich als Linearkombination $\alpha_1 + \dots + \alpha_k$ mit $\alpha_i \in \Delta$ so schreiben, dass für alle $i \in \{1,\dots,k\}$ die Partialsumme $\alpha_1 + \dots + \alpha_i$ eine Wurzel ist.
\end{cor}

\begin{proof}
  Wir führen einen Induktionsbeweis und induzieren über die Höhe $\hgt(\beta)$.
  Sei $\hgt(\beta) = 1$, dann ist $\beta \in \Delta$ und die Behauptung ist erfüllt.

  Angenommen, die Behauptung gelte für ein $k-1 \in \N$.
  Sei nun $\hgt(\beta) = k$ und damit $\beta$ positiv aber nicht einfach.
  Nach \hyperref[it:B2]{(B2)} lässt sich dann 
  \begin{displaymath}
    \beta = \alpha_1 + \dots + \alpha_k \tag{$\ast$}
  \end{displaymath}
  schreiben, wobei die $\alpha_i$ nicht zwingend paarweise verschieden sind.
  Mit Lemma \ref{lem:posRootNewRoot} folgt dann, dass ein $\alpha \in \Delta$ existiert, sodass $\beta - \alpha$ eine positive Wurzel ist.
  Aufgrund der Eindeutigkeit der Darstellung ($\ast$) bis auf Reihenfolge der Summanden, lässt sich ohne Beschränkung der Allgemeinheit $\alpha = \alpha_k$ annehmen.

  Es gilt $\hgt(\beta - \alpha_{k}) = k - 1$ und nach Induktionsvoraussetzung folgt $\alpha_1 + \dots + \alpha_{i} \in \Phi^+$ für alle $i \in \{1,\dots,k-1\}$.
  Der Fall $i = k$ folgt bereits mit ($\ast$), womit die Behauptung für $k$ vollständig bewiesen ist.
\end{proof}

\begin{lem}
  \label{lem:permute}
  Sei $\alpha \in \Delta$. 
  Dann permutiert die Spiegelung $\sigma_\alpha$ alle von $\alpha$ verschiedenen positiven Wurzeln, es gilt also $\sigma_\alpha(\Phi^+ \setminus \{\alpha\}) = \Phi^+ \setminus \{\alpha\}$.
\end{lem}

\begin{proof}
  Sei $\beta \in \Phi^+ \setminus \{\alpha\}$. 
  Dann existiert eine Darstellung der Form $\beta = \sum_{\gamma \in \Delta} k_\gamma \gamma$ mit positiven Koeffizienten $k_\gamma$.
  Es kann $\beta$ nicht proportional zu $\alpha$ sein, da sonst nach \hyperref[it:R2]{(R2)} und unter der Voraussetzung $\beta \neq \alpha$ ein Widerspruch zur Voraussetzung entsteht, dass $\beta$ positiv ist.
  Daher existiert ein $\gamma \neq \alpha$ mit $k_\gamma > 0$.

  Aufgrund der Linearität von $\sigma_\alpha$ besitzt dann auch die Wurzel $\sigma_\alpha(\beta)$ mindestens einen positiven Linearfaktor und damit ist nach \hyperref[it:B2]{(B2)} auch $\sigma_\alpha(\beta)$ positiv.
  Des Weiteren ist $\sigma_\alpha(\beta)$ nicht proportional zu $\alpha$, da nach \hyperref[it:R2]{(R2)} sonst $\beta = -\alpha$ gelten würde, was im vorangehenden Absatz bereits ausgeschlossen wurde.
  Hieraus folgt die Behauptung.
\end{proof}

\begin{cor}
  \label{cor:sigmaDelta}
  Sei $\delta := \tfrac{1}{2} \sum_{\beta \in \Phi^+} \beta$.
  Dann gilt $\sigma_\alpha(\delta) = \delta - \alpha$ für alle $\alpha \in \Delta$.
\end{cor}

\begin{proof}
  Es gilt mit Lemma \ref{lem:permute}
  \begin{displaymath}
    \frac{1}{2} \sum_{\substack{\beta \in \Phi^+\\ \beta \neq \alpha}} \sigma_\alpha(\beta) + \frac{1}{2}\alpha 
  = \frac{1}{2} \sum_{\substack{\beta \in \Phi^+\\ \beta \neq \alpha}} \beta + \frac{1}{2}\alpha
  = \frac{1}{2} \sum_{\beta \in \Phi^+} \beta
  = \delta.
  \end{displaymath}
  Daraus folgt 
  \begin{align*}
    \sigma_\alpha(\delta) 
    &= \sigma_\alpha(\frac{1}{2} \sum_{\substack{\beta \in \Phi^+\\ \beta \neq \alpha}} \sigma_\alpha(\beta) + \frac{1}{2}\alpha ) \\
    &= \frac{1}{2} \sum_{\substack{\beta \in \Phi^+\\ \beta \neq \alpha}} \sigma_\alpha(\beta) + \frac{1}{2} \sigma_\alpha(\alpha) \\
    &= \frac{1}{2} \sum_{\substack{\beta \in \Phi^+\\ \beta \neq \alpha}} \sigma_\alpha(\beta) + \frac{1}{2} \sigma_\alpha(\alpha) + \frac{1}{2} \alpha - \frac{1}{2} \alpha \\
    &= \frac{1}{2} \sum_{\substack{\beta \in \Phi^+\\ \beta \neq \alpha}} \beta + \frac{1}{2} \alpha - \alpha \\
    &= \frac{1}{2} \sum_{\beta \in \Phi^+} \beta - \alpha \\
    &= \delta - \alpha. \qedhere
  \end{align*}
\end{proof}

Nach Definition \ref{def:weylgroup} wird die \weyl\hyp{}Gruppe $\W$ von den Spiegelungen an den zu Wurzeln orthogonalen Hyperebenen erzeugt.
Das folgende Lemma liefert ein Kriterium dafür, wann man entsprechende Darstellungen von Elementen aus $\W$ vereinfachen kann.

\begin{lem}
  \label{lem:simplify}
  Seien $\alpha_1,\dots,\alpha_t$ nicht notwendig verschiedene einfache Wurzeln. Es sei $\sigma_i := \sigma_{\alpha_i}$.
  Ist $\sigma_1 \dots \sigma_{t-1}(\alpha_t)$ eine negative Wurzel, dann existiert ein Index $1 \leq s < t$, sodass $\sigma_1 \dots \sigma_t = \sigma_1 \dots \sigma_{s-1} \sigma_{s+1} \dots \sigma_{t-1}$ gilt.
\end{lem}

\begin{proof}
  Sei $\beta_i := \sigma_{i + 1} \dots \sigma_{t-1}(\alpha_t)$, für alle $0 \leq i \leq t - 2$ und $\beta_{t - 1} := \alpha_t$.
  Nach Voraussetzung ist $\beta_0$ negativ und $\beta_{t - 1}$ als einfache Wurzel positiv.
  Es existiert also aufgrund der Endlichkeit der Folge der $\beta_i$ ein minimaler Index $s$, sodass jeweils $\beta_s$ positiv und $\sigma_s(\beta_s) = \beta_{s - 1}$ negativ sind.
  Wiederum impliziert Lemma \ref{lem:permute} die Gleichung $\beta_s = \alpha_s$, da sonst andernfalls $\sigma_s\beta_s$ positiv sein müsste.
  Mit Lemma \ref{lem:conjReflection} folgt sodann
  \begin{displaymath}
    \sigma_s 
    = \sigma_{\alpha_s} 
    = \sigma_{\sigma_{s+1} \dots \sigma_{t - 1}(\alpha_t)}
    = (\sigma_{s+1} \dots \sigma_{t - 1}) \sigma_{t} (\sigma_{s+1} \dots \sigma_{t - 1})^{-1}.
  \end{displaymath}
  Dies impliziert wiederum
  \begin{align*}
    \sigma_1 \dots \sigma_t 
    &= \sigma_1 \dots \sigma_{s - 1} \sigma_s \sigma_{s+1} \dots \sigma_t \\
    &= \sigma_1 \dots \sigma_{s - 1} (\sigma_{s+1} \dots \sigma_{t - 1}) \sigma_{t} (\sigma_{s+1} \dots \sigma_{t - 1})^{-1} \sigma_{s+1} \dots \sigma_{t - 1} \sigma_t\\
    &= \sigma_1 \dots \sigma_{s - 1} (\sigma_{s+1} \dots \sigma_{t - 1}) \sigma_t  \sigma_t\\
    &= \sigma_1 \dots \sigma_{s - 1} \sigma_{s+1} \dots \sigma_{t - 1}. \qedhere
  \end{align*}
\end{proof}

\begin{cor}
  \label{cor:minimalReflection}
  Sei $\sigma = \sigma_1 \dots \sigma_t \in \W$, mit einfachen Spiegelungen $\sigma_i$, wobei $t$ minimal gewählt sei. Dann ist $\sigma(\alpha_t)$ eine negative Wurzel.
\end{cor}

\begin{proof}
  Nach Voraussetzung lässt sich der Ausdruck für $\sigma$ nicht weiter verkürzen.
  Mit Lemma \ref{lem:simplify} folgt, dass $\sigma_1 \dots \sigma_{t-1}(\alpha_t)$ positiv ist.
  Da $\sigma_t$ eine Spiegelung ist, gilt $\alpha_t = \sigma_t(-\alpha_t)$.
  Damit folgt dann, dass auch $\sigma_1 \dots \sigma_{t-1}(\sigma_t(-\alpha_t)) = \sigma_1 \dots \sigma_t(-\alpha_t)$ positiv ist.
  Die Linearität von Spiegelungen impliziert dann die Negativität von $\sigma = \sigma_1 \dots \sigma_t(\alpha_t)$.
\end{proof}
