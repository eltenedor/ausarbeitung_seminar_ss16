\section*{Einleitung}\index{Einleitung}
\addcontentsline{toc}{section}{Einleitung}\index{Einleitung}

Basierend auf \cite[S.49-55]{humphreys1972introduction} beschäftigt sich diese Ausarbeitung mit Wurzelsystemen.
Diese spielen eine Rolle bei der Klassifikation von \lie\hyp{}Algebren.

In Abschnitt \ref{sec:grundlagen} werden zunächst die für den Rest der Ausarbeitung wichtigen Grundlagen behandelt.
Auch die für die späteren Abschnitte zentrale \weyl\hyp{}Gruppe wird definiert und einige erste Eigenschaften der von ihr induzierten Gruppenoperationen beschrieben.

Abschnitt \ref{sec:einfach} beschäftigt sich mit den Erzeugern eines Wurzelsystems: den einfachen Wurzeln. 
Hier werden Eigenschaften dieser Erzeugermenge besprochen, die für die Folgeabschnitte relevant sind.

Der \hyperref[sec:weylgroup]{dritte} Teil der Ausarbeitung liefert eine genauere Beschreibung der von der \weyl\hyp{}Gruppe induzierten Gruppenoperationen und knüpft somit unter Verwendung der Ergebnisse des zweiten Abschnittes an den ersten Abschnitt an.

Im \hyperref[sec:irreduc]{letzten} Abschnitt beschäftigt sich diese Ausarbeitung mit irreduziblen Wurzelsystemen. 
Diese bilden die Grundlage für die Klassifikation von \lie\hyp{}Algebren.
