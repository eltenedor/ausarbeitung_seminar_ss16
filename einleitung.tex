\section*{Einleitung}\index{Einleitung}
\addcontentsline{toc}{section}{Einleitung}\index{Einleitung}

Bei \lie\hyp{}Algebren handelt es sich um Vektorräume, die über eine nicht assoziative Multiplikation verfügen.
Sie tauchen beispielsweise im Rahmen der Untersuchung von \lie\hyp{}Gruppen auf.
Übergeordnetes Ziel dieses Seminars ist es, komplexe halbeinfache \lie\hyp{}Algebren vollständig zu klassifizieren.
Basierend auf \cite[S.49-55]{humphreys1972introduction} beschäftigt sich diese Ausarbeitung mit Wurzelsystemen.

Konkret treten Wurzelsysteme in der Theorie der halbeinfachen \lie\hyp{}Algebren als Teilmengen des Dualraums sogenannter \emph{torischer} \lie\hyp{}Unteralgebren auf.
Es lässt sich zeigen, dass Wurzelsysteme unter Einschränkung des Skalarenkörpers reelle Untervektorräume des komplexen Dualraums erzeugen.
Eine Übertragung der auf torischen Unteralgebren nicht ausgearteten \killing\hyp{}Form auf den Dualraum der torischen Unteralgebra liefert eine Bilinearform.
Schränkt man diese auf den von den Wurzeln aufgespannten reellen Unterraum ein, wird dieser Untervektorraum zu einem \euklid ischen Vektorraum.

Andererseits lassen sich derartig erzeugte Unterräume auch losgelöst von \lie\hyp{}Algebren behandeln.
Diese Ausarbeitung beschäftigt sich daher mit den Eigenschaften abstrakter Wurzelsysteme.
Ziel dieser Ausarbeitung ist es, Eigenschaften abstrakter Wurzelsysteme darzustellen und damit die vollständige Klassifikation irreduzibler Wurzelsysteme vorzubereiten.

In Abschnitt \ref{sec:grundlagen} werden zunächst die für den Rest der Ausarbeitung wichtigen Grundlagen zu Wurzelsystemen behandelt.

Abschnitt \ref{sec:einfach} beschäftigt sich mit den Erzeugern eines Wurzelsystems: den einfachen Wurzeln. 
Hier werden Eigenschaften dieser Erzeugermenge besprochen, die für die Folgeabschnitte relevant sind.

Der \hyperref[sec:weylgroup]{dritte} Teil der Ausarbeitung liefert eine genauere Beschreibung der von der \weyl\hyp{}Gruppe induzierten Gruppenoperationen.

Im \hyperref[sec:irreduc]{letzten} Abschnitt beschäftigt sich diese Ausarbeitung mit irreduziblen Wurzelsystemen. 
Diese bilden den Ausgangspunkt für die in späteren Vorträgen beschriebene Klassifikation der Wurzelsysteme.
