\section{Die Weyl-Gruppe}
\label{sec:weylgroup}

Nach der in den bisherigen Abschnitten geleisteten Vorarbeit, wenden wir uns in diesem Abschnitt nun wieder der Gruppenoperation der \weyl\hyp{}Gruppe auf der Menge der \weyl\hyp{}Kammern beziehungsweise auf der Menge der Fundamentalsysteme eines Wurzelsystems $\Phi$ aus Proposition \ref{prop:groupOp} zu.

Wir beginnen mit einer für beliebige Gruppenoperationen definierten Eigenschaft.

\begin{defn}
  Eine Gruppe $G$ operiere auf einer Menge $X$.
  Die Gruppenoperation $\circ \colon G \times X \to X$ heißt \emph{scharf transitiv}, wenn für alle $x,y \in X$ genau ein $g \in G$ existiert, sodass $g \circ x = y$ gilt. 
\end{defn}

Ziel dieses Abschnittes ist es zu beweisen, dass die durch die Gruppe $\W$ auf der Menge der Fundamentalsysteme induzierte Gruppenoperation scharf transitiv ist.
Dies wird unter anderem auch durch den nachfolgenden Satz bewiesen.
Ein Lemma stellt zuvor noch sicher, dass es sich bei $\W$ um eine endliche Gruppe handelt.

\begin{lem}
  \label{lem:weylFinite}
  Sei $\Phi$ ein Wurzelsystem über $E$. Dann ist die entsprechende \weyl\hyp{}Gruppe endlich. 
\end{lem}

\begin{proof}
  Aus \hyperref[it:R3]{(R3)} folgt, dass $\W$ auf der Menge aller Wurzeln operiert, welche nach \hyperref[it:R1]{(R1)} endlich ist.
  Es existiert also ein Gruppenhomomorphismus in die symmetrische Gruppe $\Sym(\Phi)$.

  Ist dieser Homomorphismus injektiv, also die Gruppenoperation treu, so lässt sich $\W$ als eine Untergruppe von $\Sym(\Phi)$ auffassen und ist damit endlich.
  Sei $\sigma \in \W$ im Kern dieses Homomorphismus.
  Dann fixiert $\sigma$ alle Wurzeln aus $\Phi$.
  Da nach \hyperref[it:R1]{(R1)} die Wurzeln den Vektorraum $E$ aufspannen ist $\sigma$ bereits eindeutig festgelegt und es folgt $\sigma = \mathds{1}$.
  Also ist die Gruppenoperation treu und $\W$ damit endlich.
\end{proof}


\begin{thm}
  \label{thm:simplyTransitive}
  Es sei $\Delta$ ein Fundamentalsystem des Wurzelsystems $\Phi$ über $E$ und zugehöriger \weyl\hyp{}Gruppe $\W$.
  Dann gelten die folgenden Aussagen:
  \begin{enumerate}[(a)]

    \item Wenn $\gamma \in E$ regulär ist, dann existiert ein $\sigma \in \W$, sodass für alle $\alpha \in \Delta$ die Ungleichung $(\sigma(\gamma), \alpha) > 0$ gilt. 
      Es operiert $\W$ also transitiv auf der Menge der \weyl\hyp{}Kammern.

    \item Wenn $\Delta'$ ein weiteres Fundamentalsystem von $\Phi$ ist, dann existiert ein $\sigma \in \W$, sodass $\sigma(\Delta') = \Delta$.
      Es operiert $\W$ also transitiv auf der Menge der Fundamentalsysteme.

    \item Für alle $\alpha \in \Phi$ existiert ein $\sigma \in \W$ mit $\sigma(\alpha) \in \Delta$.
      Jede Wurzel liegt also in der $\W$\hyp{}Bahn einer einfachen Wurzel.
    \item Die \weyl\hyp{}Gruppe $\W$ wird erzeugt von den Spiegelungen $\sigma_\alpha$ für $\alpha \in \Delta$.

    \item Ist $\sigma \in \W$ und gilt $\sigma(\Delta) = \Delta$, so folgt $\sigma = \mathds{1}$.
      Also operiert $\W$ scharf transitiv auf der Menge der Fundamentalsysteme.
  \end{enumerate}
\end{thm}

\begin{proof}
  Wir führen die Beweise für (a) bis (c) zunächst für die von den einfachen Spiegelungen erzeugte Untergruppe $\W'$ von $\W$.
  Nach dem Beweis von (d) folgt $\W' = \W$.

  (a):
  Es sei $\gamma \in E$ regulär und $\delta := \tfrac{1}{2} \sum_{\alpha \in \Phi^+} \alpha$.
  Da nach Lemma \ref{lem:weylFinite} die \weyl\hyp{}Gruppe und damit auch die Untergruppe $\W'$ endlich sind, existiert ein $\sigma \in \W'$, sodass $(\sigma(\gamma), \delta)$ maximal wird.
  Für alle $\alpha \in \Delta$ gilt auch $\sigma_\alpha\sigma \in \W'$ aufgrund der Untergruppeneigenschaft.
  Daraus folgt
  \begin{align*}
    (\sigma(\gamma), \delta) 
    &\geq (\sigma_\alpha \sigma(\gamma), \delta) &&\text{aufgrund der Maximalitätseigenschaft von $(\sigma(\gamma),\delta)$} \\
    &= (\sigma(\gamma), \sigma_\alpha(\delta)) &&\text{da $\sigma_\alpha$ eine Isometrie und zudem selbstinvers ist} \\
    &= (\sigma(\gamma), \delta - \alpha) &&\text{nach Korollar \ref{cor:sigmaDelta}} \\
    &= (\sigma(\gamma), \delta) - (\sigma(\gamma), \alpha) &&\text{aufgrund der Bilinearität des Skalarproduktes.}
  \end{align*}
  Also muss $(\sigma(\gamma), \alpha) \geq 0$ für alle $\alpha \in \Delta$ gelten, um nicht der Maximalität von $\sigma$ zu widersprechen.

  Nach Voraussetzung ist $\gamma$ regulär.
  Daher kann für kein $\alpha \in \Delta$ die Identität $(\sigma(\gamma),\alpha) = 0$ gelten, da sonst $\gamma$ in $P_{\sigma^{-1}(\alpha)}$ liegt, was der vorausgesetzten Regularität widerspricht.
  Somit gilt für alle $\alpha \in \Delta$ die strikte Ungleichung $(\sigma(\gamma), \alpha) > 0$.
  Das bedeutet jedoch gerade, dass $\sigma(\gamma)$ in der Fundamentalkammer $\mathfrak{C}(\Delta)$ liegt.

  Also bildet die einfache Spiegelung $\sigma$ entsprechend Proposition \ref{prop:groupOp} die \weyl\hyp{}Kammer $\mathfrak{C}(\gamma)$ auf $\mathfrak{C}(\Delta)$ ab. 
  Somit gilt, dass alle \weyl\hyp{}Kammern mit der Fundamentalkammer verbunden sind, die Menge der \weyl\hyp{}Kammern ist also eine $\W'$\hyp{}Bahn und die Gruppenoperation damit transitiv.

  (b):
  Da nach dem vorangehenden Beweis $\W'$ die \weyl\hyp{}Kammern transitiv permutiert, gilt dies nach Proposition \ref{prop:groupOp} auch für die korrespondierende Gruppenoperation auf Fundamentalsystemen.

  (c):
  Da nach (b) die Untergruppe $\W'$ transitiv auf der Menge der Fundamentalsysteme operiert, genügt es nachzuweisen, dass jede Wurzel $\alpha \in \Phi$ in einem Fundamentalsystem enthalten ist.
  Aufgrund von Axiom \hyperref[it:R2]{(R2)} sind $\pm \alpha$ die einzigen zu $\alpha$ proportionalen Wurzeln .
  Somit unterscheiden sich alle Hyperebenen $P_\beta$ von $P_\alpha = P_{-\alpha}$.
  Nach Proposition \ref{prop:unterraumAusschoepfung} lässt sich $P_\alpha$ nicht von den Unterräumen $P_\alpha \cap P_\beta$ mit $\alpha \neq \beta$ ausschöpfen.
  Es existiert daher ein $\gamma \in P_\alpha$ mit $\gamma \not\in P_\beta$ für alle $\beta \neq \alpha$.

  Entsprechend Satz \ref{thm:base} korrespondiert zu jedem regulären Element ein Fundamentalsystem. Es gilt nun in der Nähe von $\gamma$ ein geeignetes reguläres Element zu finden und nachzuweisen, dass $\alpha$ in dem entsprechenden Fundamentalsystem enthalten ist.

  Wir wählen nun ein $\varepsilon > 0$ und $\gamma' \in E$ möglichst nahe bei $\gamma$, sodass $(\gamma', \alpha) = \varepsilon > 0$ ist und $|(\gamma', \beta)| > \varepsilon$ für alle $\beta \neq \pm \alpha$ gilt ($\ast$).
  Dies ist möglich, da nach Konstruktion $(\gamma, \beta) \neq 0$ für alle $\beta \neq \pm \alpha$ gilt.
  Damit existiert aufgrund der Stetigkeit des Skalarproduktes und der Betragsfunktion ein $\delta > 0$, sodass für alle $\gamma' \in B_\delta(\gamma)$ auch $\varepsilon < |(\gamma', \beta)|$ gilt, falls man $\varepsilon = \tfrac{1}{2} \min_{\beta \neq \pm\alpha}{|(\gamma,\beta)|}$ wählt.
  Da die Hyperebene $P_\alpha$ nicht offen ist, existiert also auch ein $\gamma' \in P_\alpha \setminus B_\delta(\gamma)$ und wir können durch Verkleinerung von $\varepsilon$ und eventuelle Multiplikation von $\gamma'$ mit $-1$ annehmen, dass $(\gamma', \alpha) = \varepsilon$ gilt.

  Nach Konstruktion ist $\gamma'$ regulär.
  Zudem ist $\alpha$ unzerlegbar.
  Denn wäre $\alpha$ zerlegbar, so existieren $\beta_1, \beta_2 \in \Phi^+(\gamma')$ mit $\alpha = \beta_1 + \beta_2$.
  Mit der Bilinearität des Skalarproduktes folgt dann jedoch unter Berücksichtigung von ($\ast$)
  \begin{displaymath}
    \varepsilon 
    = (\gamma', \alpha) 
    = (\gamma', \beta_1 + \beta_2) 
    = (\gamma', \beta_1) + (\gamma', \beta_2)
    \geq \varepsilon + \varepsilon,
  \end{displaymath}
  also ein Widerspruch.
  Folglich ist $\alpha$ unzerlegbar, was $\alpha \in \Delta(\gamma')$ impliziert.

  (d):
  Da nach Konstruktion $\W'$ eine Untergruppe von $\W$ ist, reicht es zu zeigen, dass jede Spiegelung $\sigma_\alpha$ mit $\alpha \in \Phi$ ein Element von $\W'$ ist.
  Nach (c) existiert ein $\sigma \in \W'$, sodass $\beta := \sigma(\alpha) \in \Delta$.
  Damit gilt insbesondere $\sigma_\beta \in \W'$.
  Aufgrund von Lemma \ref{lem:conjReflection} ist $\sigma_{\sigma(\alpha)} = \sigma \sigma_\alpha \sigma^{-1}$, also folgt mit der Untergruppeneigenschaft von $\W'$ sodann $\sigma_\alpha = \sigma^{-1} \sigma_\beta \sigma \in \W'$.
  Folglich gilt $\W' = \W$ und alle bereits für $\W'$ bewiesenen Aussagen gelten auch für $\W$.

  (e):
  Sei $\sigma \in \W$ mit $\sigma(\Delta) = \Delta$.
  Nehmen wir $\sigma \neq \mathds{1}$ an so lässt sich nach (d) die Abbildung $\sigma$ als Produkt einfacher Spiegelungen schreiben, es gilt also $\sigma = \sigma_1\dots\sigma_t$ und $\sigma_1 \dots \sigma_t(\alpha_t) \preceq 0$. 
  Unter Berücksichtigung von Lemma \ref{lem:simplify} kann man annehmen, dass die multiplikative Darstellung nicht mehr verkürzt werden kann.
  Dann ist nach Korollar \ref{cor:minimalReflection} das Bild $\sigma(\alpha_t)$ negativ, also gilt insbesondere $\sigma(\alpha_t) \not\in \Delta$ im Widerspruch zur Voraussetzung.
  Die Annahme muss also verworfen werden und es folgt $\sigma = \mathds{1}$.
\end{proof}

Aus der Tatsache, dass $\W$ von einfachen Spiegelungen erzeugt wird, lassen sich nun weitere Aussagen folgern.

\begin{defn}
  Sei $\Phi$ ein Wurzelsystem mit Fundamentalsystem $\Delta$ und \weyl\hyp{}Gruppe $\W$. 
  Des Weiteren sei $\sigma \in \W$, mit einer Darstellung $\sigma = \sigma_{\alpha_1}\dots\sigma_{\alpha_t}$ ($\ast$), wobei $t$ minimal gewählt sei und $\alpha_i \in \Delta$ gelte. 
  Dann nennt man den Ausdruck ($\ast$) auch \emph{reduziert} und definiert die \emph{Länge von $\sigma$ bezüglich} $\Delta$ durch $\ell(\sigma) := t$.
\end{defn}

\begin{lem}
  \label{lem:lengthAndNegativeRoots}
  Sei $\Phi$ ein Wurzelsystem in $E$ mit Fundamentalsystem $\Delta$ und $\W$ die zugehörige \weyl\hyp{}Gruppe.
  Für $\sigma \in \W$ bezeichne $n(\sigma)$ die Anzahl der positiven Wurzeln $\alpha$ mit negativem Bild $\sigma(\alpha)$.
  Dann gelten die folgenden Aussagen:
  \begin{enumerate}[(1)]
    \item Sei $\alpha \in \Delta$. Falls $\sigma(\alpha)$ negativ ist, so gilt $n(\sigma\sigma_\alpha(\alpha)) = n(\sigma) - 1$.
    \item Es gilt $\ell(\sigma) = n(\sigma)$.
  \end{enumerate}
\end{lem}

\begin{proof}
  (1):
  Sei $\sigma(\alpha)$ negativ.
  Aus Lemma \ref{lem:permute} folgt, dass $\sigma_\alpha$ alle von $\alpha$ verschiedenen positiven Wurzeln permutiert.
  Einerseits enthält $\sigma(\Phi^+ \setminus \{\alpha\})$ somit genauso viele negative Wurzeln wie $\sigma(\sigma_\alpha(\Phi^+ \setminus \{\alpha\}))$.
  Andererseits ist jedoch 
  $\sigma \sigma_\alpha (\alpha) = \sigma(-\alpha) = -\sigma(\alpha)$ 
  positiv.
  Daraus können wir $n(\sigma\sigma_\alpha) = n(\sigma) - 1$ folgern.

  (2):
  Wir führen den Beweis mittels Induktion über $\ell(\sigma)$.

  Ist $\ell(\sigma)=0$, so gilt $\sigma = \mathds{1}$, was wiederum $n(\sigma)=0$ zur Folge hat.

  Angenommen, die Behauptung gelte für alle $\tau \in \W$ mit $\ell(\tau) < \ell(\sigma)$.
  Wir schreiben nun $\sigma$ in der reduzierten Form $\sigma = \sigma_{\alpha_1} \dots \sigma_{\alpha_t}$ und definieren $\alpha := \alpha_t$.
  Über Korollar \ref{cor:minimalReflection} folgt damit die Negativität von $\sigma(\alpha)$.
  Aufgrund von (1) können wir einerseits $n(\sigma\sigma_\alpha) = n(\sigma) - 1$ folgern.
  
  Andererseits gilt $\ell(\sigma\sigma_\alpha) = \ell(\sigma_{\alpha_1} \dots \sigma_{\alpha_{t -1}}) = \ell(\sigma) - 1 < \ell(\sigma)$.
  Mit der Induktionsvoraussetzung folgt somit $\ell(\sigma\sigma_\alpha) = n(\sigma\sigma_\alpha)$, also $\ell(\sigma) - 1 = n(\sigma) - 1$, woraus sich die Behauptung ergibt.
\end{proof}

Nun betrachten wir die einfach transitive Operation der \weyl\hyp{}Gruppe auf den \weyl\hyp{}Kammern aus Satz \ref{thm:simplyTransitive}.
Es soll gezeigt werden, dass der in der \euklid ischen Topologie gebildete Abschluss $\overline{\mathfrak{C}(\Delta)}$ der Fundamentalkammer bezüglich des Fundamentalsystems $\Delta$ ein \emph{Fundamentalbereich} der Gruppenoperation von $\W$ auf $E$ ist.
Dies bedeutet, dass für alle Vektoren $x \in E$ genau ein $y \in \overline{\mathfrak{C}(\Delta)}$ existiert mit $x \in \W y = \{\sigma(y) \mid \sigma \in \W\}$.

\begin{lem}
  \label{lem:fundamentalDomain}
  Sei $\Phi$ ein Wurzelsystem in $E$ mit Fundamentalsystem $\Delta$ und $\W$ die zugehörige \weyl\hyp{}Gruppe.
  Dann gelten folgende Aussagen:
  \begin{enumerate}[(1)]
    \item Jeder Punkt aus $E$ liegt in der $\W$\hyp{}Bahn eines Elementes aus  $\overline{\mathfrak{C}(\Delta)}$.
    \item Falls für $\lambda, \mu \in \overline{\mathfrak{C}(\Delta)}$ ein $\sigma \in \W$ existiert mit $\sigma \lambda = \mu$, dann ist $\sigma$ das Produkt einfacher Spiegelungen, die $\lambda$ fixieren, und es gilt speziell $\lambda = \mu$.
  \end{enumerate}
  Insbesondere ist $\overline{\mathfrak{C}(\Delta)}$ ein Fundamentalbereich der Gruppenoperation der Gruppe $\W$ auf dem Vektorraum $E$.
\end{lem}

\begin{proof}
  (1):
  Wir setzen die in Abschnitt \ref{sec:grundlagen} auf der Menge der Wurzeln $\Phi$ definierte Halbordnungsrelation $\succeq$ nun auf ganz $E$ fort, indem wir genau dann $\beta \preceq \alpha$ setzen, wenn $\alpha - \beta$ eine $\R$-Linearkombination mit sämtlich positiven Koeffizienten ist oder $\alpha = \beta$ gilt.

  Sei $\lambda \in E$ gegeben. 
  Wir betrachten als nächstes die Teilmenge $M$ aller $\W$\hyp{}Bahnelemente $\mu \in \W\lambda$, für die $\mu \succeq \lambda$ gilt.
  Da $M$ endlich und aufgrund von $\lambda \in M$ nichtleer ist, existieren bezüglich der Halbordnung $\succeq$ maximale Elemente.

  Sei $\mu = \sigma\lambda$ mit $\sigma \in \W$ ein solches maximales Element.
  Wir möchten nun zeigen, dass $\mu \in \overline{\mathfrak{C}(\Delta)}$ gilt.
  Sei dazu $\alpha \in \Delta$.
  Es ist $\sigma_\alpha(\mu) = \mu - \langle \mu, \alpha \rangle \alpha$.
  Da $\sigma_\alpha(\mu) = \sigma_\alpha \sigma \lambda \in \W\lambda$ muss, um nicht der Maximalität von $\mu$ zu widersprechen $\langle \mu, \alpha \rangle \geq 0$ gelten.
  Damit liegt jedoch $\mu$ in $\overline{\mathfrak{C}(\Delta)}$.

  (2):
  Wir führen einen Induktionsbeweis über die Länge $\ell(\sigma)$.

  Ist $\ell(\sigma) = 0$, so folgt $\sigma = \mathds{1}$, womit sich $\lambda = \mu$ ergibt.

  Sei nun $\ell(\sigma) > 0$ und die Aussage für alle $\tau \in \W$ mit $\ell(\tau) < \ell(\sigma)$ bereits bewiesen.
  Nach Lemma \ref{lem:lengthAndNegativeRoots} bildet $\sigma$ eine positive Wurzel auf eine negative Wurzel ab.
  Insbesondere existiert also eine einfache Wurzel $\alpha \in \Delta$ mit negativem Bild $\sigma(\alpha)$.
  Für alle $\xi \in \mathfrak{C}(\Delta)$ gilt $(\xi, \sigma(\alpha)) = \sum_{\beta \in \Delta} k_\beta (\xi,\beta) < 0$. 
  Denn alle $k_\beta$ sind nach Voraussetzung negativ.  
  Zudem gilt $(\xi,\beta) > 0$, weil $\xi$ als Element der Fundamentalkammer mit allen einfachen Wurzeln einen spitzen Winkel einschließt.
  Aufgrund der Stetigkeit des Skalarproduktes ergibt sich somit auch $0 \geq (\mu, \sigma(\alpha))$.
  
  Andererseits folgt mit derselben Argumentation angewendet auf $\lambda$ die Identität
  \begin{displaymath}
    0 \geq (\mu, \sigma(\alpha)) = (\sigma^{-1}(\mu), \alpha) = (\lambda, \alpha) \geq 0,
  \end{displaymath}
  weil Spiegelungen Isometrien sind.
  Hieraus folgt zunächst $(\lambda, \alpha) = 0$, also gilt $\lambda \in P_\alpha$.
  Dies wiederum impliziert $\sigma_\alpha(\lambda) = \lambda$ und weiter noch $\sigma \sigma_\alpha(\lambda) = \mu$.
  Somit erfüllt $\sigma \sigma_\alpha$ die Voraussetzungen des zu beweisenden Lemmas. 

  Wir zeigen nun, dass auf $\sigma \sigma_\alpha$ auch die Induktionsannahme zutrifft.
  Nach Lemma \ref{lem:lengthAndNegativeRoots}(1) gilt $n(\sigma\sigma_\alpha) = n(\sigma) - 1$, woraus schließlich mit \ref{lem:lengthAndNegativeRoots}(2) die Gleichheit $\ell(\sigma\sigma_\alpha) = \ell(\sigma) - 1$ folgt.
  Es gilt also $\ell(\sigma\sigma_\alpha) < \ell(\sigma)$ und darum lässt sich aufgrund der Induktionsannahme $\lambda = \mu$ folgern.
\end{proof}
 
Eine Verschärfung von Lemma \ref{lem:fundamentalDomain} findet sich in \cite[S.22]{humphreys1992reflection}.
