\section{Die Weyl-Gruppe}
\label{sec:weylgroup}

Nach der in den bisherigen Abschnitten geleisteten Vorarbeit, wenden wir uns in diesem Abschnitt nun wieder der Gruppenoperation der \weyl\hyp{}Gruppe auf der Menge der\weyl\hyp{}Kammern beziehungsweise der Menge der Fundamentalsysteme eines Wurzelsystems $\Phi$ aus Proposition \ref{prop:groupOp} zu.

Wir beginnen mit einer für beliebige Gruppenoperationen definierten Eigenschaft.

\begin{defn}
  Eine Gruppe $G$ operiere auf einer Menge $X$.
  Die Gruppenoperation $\circ \colon G \times X \to X$ heißt \emph{scharf transitiv}, wenn für alle $x,y \in X$ genau ein $g \in G$ existiert, sodass $g \circ x = y$. 
\end{defn}

Ziel dieses Abschnittes ist es nun zu beweisen, dass die durch $\W$ auf der Menge der Fundamentalsysteme induzierte Gruppenoperation scharf transitiv ist.
Dies wird unter anderem auch durch den nachfolgenden Satz bewiesen.
Ein Lemma stellt zuvor noch Sicher, dass es sich bei $\W$ um eine endliche Gruppe handelt.

\begin{lem}
  Es sei $\Phi$ ein Wurzelsystem in $E$. Dann ist die entsprechende \weyl\hyp{}Gruppe endlich. 
\end{lem}

\begin{proof}
  Wie mit den (R3) \ref{prop:groupOp} folgt, operiert $\W$ auf der Menge aller Wurzeln, welche nach (R1) endlich ist.
  Es existiert also ein Gruppenhomomorphismus in die symmetrische Gruppe $\Sym(\Phi)$.

  Ist dieser Homomorphismus injektiv, also die Gruppenoperation treu, so lässt sich $\W$ als eine Untergruppe von $\Sym(\Phi)$ auffassen und ist damit endlich.
  Sei $\sigma \in \W$ im Kern dieses Homomorphismus.
  Dann fixiert $\sigma$ alle Wurzeln aus $\Phi$.
  Da nach (R1) die Wurzeln den Vektorraum $E$ aufspannen ist $\sigma$ bereits eindeutig festgelegt und es folgt $\sigma = id$.
  Also ist die Gruppenoperation treu und $\W$ damit endlich.
\end{proof}


\begin{thm}
  Es sei $\Delta$ ein Fundamentalsystem des Wurzelsystems $\Phi$ in $E$.
  Dann gelten die folgenden Aussagen:
  \begin{enumerate}[(a)]

    \item Wenn $\gamma \in E$ regulär ist, dann existiert ein $\sigma \in \W$, sodass $(\sigma(\gamma), \alpha) > 0$ für alle $\alpha \in \Delta$. 
      Es operiert $\W$ also transitiv auf der Menge der \weyl\hyp{}Kammern.

    \item Wenn $\Delta'$ ein weiteres Fundamentalsystem von $\Phi$ ist, dann existiert ein $\sigma \in \W$, sodass $\sigma(\Delta') = \Delta$.
      Es operiert $\W$ also transitiv auf der Menge der Fundamentalsysteme.

    \item Für eine beliebige Wurzel $\alpha \in \Phi$ existiert ein $\sigma \in \W$, sodass $\sigma(\alpha) \in \Delta$.
      Jede Wurzel liegt also in der Bahn einer einfachen Wurzel.
    \item Die \weyl\hyp{}Gruppe wird erzeugt von den Spiegelungen $\sigma_\alpha$ für $\alpha \in \Delta$.

    \item Ist $\sigma \in \W$ und gilt $\sigma(\Delta) = \Delta$, so folgt $\sigma = id_{E}$.
      Also operiert $\W$ einfach transitiv auf der Menge der Fundamentalsysteme.
  \end{enumerate}
\end{thm}

\begin{proof}
  Wir führen die Beweise für (a) bis (c) zunächst für die von den einfachen Spiegelungen erzeugte Untergruppe $\W'$ von $\W$.
  Nach dem Beweis von (d) folgt $\W' = \W$.

  (a):
  Es sei $\delta = \tfrac{1}{2} \sum_{\alpha \in \Phi^+} \alpha$.
  Aufgrund 

  (b):
  Da nach dem vorangehenden Beweis $\W$ die \weyl\hyp{}Kammern transitiv permutiert, gilt dies nach Proposition \ref{prop:groupOp} auch für die korrespondierende Gruppenoperation auf Fundamentalsystemen.

  (c):

  (d):

  (e):
\end{proof}
