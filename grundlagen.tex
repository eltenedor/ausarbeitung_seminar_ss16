\section{Grundlagen zu Wurzelsystemen}

Dieser Abschnitt beinhaltet die für diese Arbeit benötigten Grundlagen zu Wurzelsystemen.

Im Folgenden bezeichne $E$ stets einen \emph{\textsc{euklid}ischen} Vektorraum, also einen $\R$-Vektorraum mit Skalarprodukt $(\cdot,\cdot)$.
Unter einer \emph{Spiegelung} $\sigma$ versteht man eine orthogonale Abbildung, welche eine \emph{Hyperebene}, also einen Unterraum der Kodimension $1$, punktweise fixiert und jeden Vektor des orthogonalen Komplements der Hyperebene auf sein Negatives abbildet.
Jeder Vektor $\alpha \in E \setminus \{0\}$ induziert eine \emph{Spiegelung} $\sigma_\alpha$ an der Hyperebene 
\begin{displaymath}
  P_\alpha := \spn(\{\alpha\})^\perp = \{\beta \in E \mid (\beta, \alpha) = 0\}.
\end{displaymath}
Definiert man nun $\langle \beta, \alpha \rangle := \tfrac{2 (\beta, \alpha)}{(\beta, \alpha)}$, so gilt
\begin{displaymath}
  \sigma_\alpha(\beta) 
  = \beta - \frac{2 (\beta, \alpha)}{(\alpha,\alpha)} \alpha 
  = \beta - \langle \beta, \alpha \rangle \alpha,
\end{displaymath}
denn $\sigma_\alpha(\alpha) = -\alpha$ und $\sigma_\alpha(\beta) = \beta$ für alle $\beta \in P_\alpha$.
Man beachte, dass im Gegensatz zum Skalarprodukt, der Ausdruck $\langle \alpha, \beta \rangle$ nur linear in der ersten Variablen ist.
Es gilt jedoch $\sign \langle \alpha, \beta \rangle = \sign (\alpha, \beta)$ für alle $\alpha, \beta \in E$.

\begin{defn}
  Eine Teilmenge $\Phi$ des euklidischen Vektorraums $E$ heißt \emph{Wurzelsystem in} $E$, falls folgende Bedingungen erfüllt sind:
  \begin{enumerate}[(R1)]
    \item Die Menge $\Phi$ ist endlich, sie spannt $E$ auf und sie enthält nicht die $0$.
    \item Falls $\alpha \in \Phi$, so sind $\pm \alpha$ die einzigen Vielfachen von $\alpha$ in $\Phi$.
    \item Falls $\alpha \in \Phi$, so lässt die Spiegelung $\sigma_\alpha$ die Menge $\Phi$ invariant, also $\sigma_\alpha(\Phi) = \Phi$.
    \item Falls $\alpha, \beta \in \Phi$, dann ist $\langle \beta, \alpha \rangle \in \Z$.
  \end{enumerate}
\end{defn}

Oft lassen sich Eigenschaften von Wurzelsystemen, bereits anhand der Eigenschaften von Erzeugern dieses Wurzelsystems ausmachen.

\begin{defn}
  Eine Teilmenge $\Delta$ von $\Phi$ heißt \emph{Fundamentalsystem}, falls die folgenden Bedingungen erfüllt sind:
  \begin{enumerate}[(B1)]
    \item Es ist $\Delta$ eine Vektorraumbasis von $E$.
    \item Jede Wurzel $\beta \in \Phi$ lässt sich schreiben als $\beta = \sum_{\alpha \in \Delta} k_\alpha \alpha$ mit ganzzahligen Linearfaktoren $k_\alpha$ die alle dasselbe Vorzeichen besitzen.
  \end{enumerate}
  Die Elemente von $\Delta$ bezeichnet man auch als $\emph{einfache}$ Wurzeln.
\end{defn}

\begin{bem}
  Einen Beweis dafür, dass jedes Fundamentalsystem eine Basis besitzt, findet man zum Beispiel in \cite[S.48]{humphreys1972introduction} oder \cite[S.116]{erdmann2006introduction}.
  Aus Eigenschaft (B1) von Fundamentalsystemen folgt sofort, dass die Linearfaktoren in (B2) eindeutig bestimmt sind. 
  Es lässt sich daher die Höhenfunktion 
  \begin{displaymath}
    \hgt(\beta) := \sum_{\alpha \in \Delta} k_\alpha 
  \end{displaymath}
  definieren.
  Entsprechend des Vorzeichens der Höhenfunktion bezeichnet man Wurzeln auch als \emph{positiv} oder \emph{negativ}.
  Einfache Wurzeln sind stets positiv.
  Zudem induziert jedes Fundamentalsystem $\Delta$ eine Halbordnung auf $\Phi$ durch
  \begin{displaymath}
    \beta \preceq \alpha \quad \text{gilt genau dann, wenn} \quad \alpha - \beta \text{ positiv ist oder } \alpha = \beta \text{ gilt}.
  \end{displaymath}
\end{bem}

Bezüglich eines Wurzelsystems $\Phi$ lassen sich auch Vektoren des umfassenden Vektorraums klassifizieren.

\begin{defn}
  Sei $\Phi$ ein Wurzelsystem in $E$.
  Ein Vektor $\gamma \in E$ heißt \emph{regulär}, falls $\gamma \in E \setminus \bigcup_{\alpha \in \Phi} P_\alpha$.
  Die Familie $(P_\alpha)_{\alpha \in \Phi}$ liefert eine Partition von $E$ in maximal zusammenhängende Mengen, die sogenannten \emph{\weyl-Kammern}.
Die jedem $\gamma \in E$ eindeutig zugeordnete \weyl-Kammer zuordnen, werde mit $\mathfrak{C}(\gamma)$ bezeichnet.
\end{defn}

Liegen zwei reguläre Wurzeln $\gamma, \gamma'$ in derselben \weyl-Kammer, so liegen sie bezüglich allen Hyperebenen $P_\alpha$ derselben Seite, was bedeutet, dass $\sign (\gamma, \alpha) = \sign( \gamma', \alpha)$ gilt, für alle $\alpha \in \Phi$.
Bezeichnet man mit
\begin{displaymath}
  \Phi^+(\gamma) := \{ \alpha \in \Phi \mid (\gamma, \alpha) > 0 \}
\end{displaymath}
die Menge aller Wurzeln, die mit $\gamma$ einen spitzen Winkel einschließen, so gilt in diesem Falle $\Phi^+(\gamma) = \Phi^+(\gamma')$.
Bezeichnet man zudem mit $\Delta(\gamma)$ das Fundamentalsystem aller Wurzeln $\alpha \in \Phi^+(\gamma)$, die sich als Summe $\alpha = \beta_1 + \beta_2$ zweier positiver Wurzeln $\beta_1, \beta_2 \in \Phi^+(\gamma)$ schreiben lassen, so gilt zusätzlich $\Delta(\gamma) = \Delta(\gamma')$. 
Dass $\Delta(\gamma)$ tatsächlich ein Fundamentalsystem von $\Phi$ ist, lässt sich in \cite[S.48]{humphreys1972introduction} nachlesen.
Wurzeln, die sich in der oben genannten Weise ausdrücken lassen, bezeichnet man auch als \emph{zerlegbar}.

Das folgende Lemma fasst nochmals die vorangehenden Überlegungen zusammen.

\begin{lem}
  Seien $\gamma, \gamma' \in E$ regulär bezüglich des Wurzelsystems $\Phi$.
  Dann folgt aus $\mathfrak{C}(\gamma) = \mathfrak{C}(\gamma')$, dass $\Phi^+(\gamma) = \Phi^+(\gamma')$. 
  Dies ist wiederum genau dann der Fall, wenn $\Delta(\gamma) = \Delta(\gamma')$ gilt.
  Jeder \weyl-Kammer $\mathfrak{C}(\gamma)$ entspricht also genau ein Fundamentalsystem $\Delta(\gamma)$.
\end{lem}

Die soeben eingeführten Begriffe veranschaulicht Abbildung \ref{fig:fundamentalWeylChamber}.

\begin{figure}
  \caption{Das Wurzelsystem A2}
  \label{fig:fundamentalWeylChamber}
\end{figure}

\begin{defn}
  Sei $\Phi$ ein Wurzelsystem in $E$ mit Fundamentalsystem $\Delta$.
  Gilt für ein reguläres $\gamma \in E$, dass $\Delta = \Delta(\gamma)$, so bezeichnet man $\mathfrak{C}(\Delta) := \mathfrak{C}(\gamma)$ als \emph{Fundamentalkammer bezüglich} $\Delta$.
\end{defn}

Wir betrachten nun einen Spezialfall von Spiegelungsgruppen. Allgemeine Spiegelungsgruppen werden in \cite{humphreys1992reflection} behandelt.

\begin{defn}
  Sei $\Phi$ ein Wurzelsystem in $E$. Dann bezeichnet $\mathcal{W}$ die von den Spiegelungen $\sigma_\alpha$, $\alpha \in \Phi$, erzeugte Untergruppe der allgemeinen linearen Gruppe $\GL(E)$. Man nennt $\mathcal{W}$ die \emph{\weyl-Gruppe} von $\Phi$.
\end{defn}

