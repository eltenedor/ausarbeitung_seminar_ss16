\section{Grundlagen zu Wurzelsystemen}

Dieser Abschnitt beinhaltet die für diese Arbeit benötigten Grundlagen zu Wurzelsystemen.

Im Folgenden bezeichne $E$ stets einen \emph{\textsc{euklid}ischen} Vektorraum, also einen $\R$\hyp{}Vektorraum mit Skalarprodukt $(\cdot,\cdot)$.
Unter einer \emph{Spiegelung} $\sigma$ versteht man eine orthogonale Abbildung, welche eine \emph{Hyperebene}, also einen Unterraum der Kodimension $1$, punktweise fixiert und jeden Vektor des orthogonalen Komplements der Hyperebene auf sein Negatives abbildet.
Jeder Vektor $\alpha \in E \setminus \{0\}$ induziert eine \emph{Spiegelung} $\sigma_\alpha$ an der Hyperebene 
\begin{displaymath}
  P_\alpha := \spn(\{\alpha\})^\perp = \{\beta \in E \mid (\beta, \alpha) = 0\}.
\end{displaymath}
Definiert man nun $\langle \beta, \alpha \rangle := \tfrac{2 (\beta, \alpha)}{(\beta, \alpha)}$, so gilt
\begin{displaymath}
  \sigma_\alpha(\beta) 
  = \beta - \frac{2 (\beta, \alpha)}{(\alpha,\alpha)} \alpha 
  = \beta - \langle \beta, \alpha \rangle \alpha,
\end{displaymath}
denn $\sigma_\alpha(\alpha) = -\alpha$ und $\sigma_\alpha(\beta) = \beta$ für alle $\beta \in P_\alpha$.
Man beachte, dass im Gegensatz zum Skalarprodukt, der Ausdruck $\langle \alpha, \beta \rangle$ nur linear in der ersten Variablen ist.
Es gilt jedoch $\sign \langle \alpha, \beta \rangle = \sign (\alpha, \beta)$ für alle $\alpha, \beta \in E$.

\begin{defn}
  Eine Teilmenge $\Phi$ des euklidischen Vektorraums $E$ heißt \emph{Wurzelsystem in} $E$, falls folgende Bedingungen erfüllt sind:
  \begin{enumerate}[(R1)]
    \item Die Menge $\Phi$ ist endlich, sie spannt $E$ auf und sie enthält nicht die $0$.
    \item Falls $\alpha \in \Phi$, so sind $\pm \alpha$ die einzigen Vielfachen von $\alpha$ in $\Phi$.
    \item Falls $\alpha \in \Phi$, so lässt die Spiegelung $\sigma_\alpha$ die Menge $\Phi$ invariant, also $\sigma_\alpha(\Phi) = \Phi$.
    \item Falls $\alpha, \beta \in \Phi$, dann ist $\langle \beta, \alpha \rangle \in \Z$.
  \end{enumerate}
\end{defn}

Wurzelsysteme sind im Allgemeinen nicht abgeschlossen unter Addition. 
Das nachfolgende Lemma beschreibt, unter welchen Bedingungen die Summe oder Differenz zweier Wurzeln wieder eine Wurzel ergibt. Der Beweis hierzu findet sich in \cite[S.45]{humphreys1972introduction}

\begin{lem}
  \label{lem:sumDiffRoot}
  Sei $\Phi$ ein Wurzelsystem von $E$ und $\alpha, \beta$ zueinander nicht proportionale Wurzeln.
  Falls $(\alpha, \beta) > 0$, dann ist $\alpha - \beta$ eine Wurzel.
  Gilt hingegen $(\alpha, \beta) < 0$, so ist $\alpha + \beta$ eine Wurzel. \qed
\end{lem}

Oft lassen sich Eigenschaften von Wurzelsystemen, bereits anhand der Eigenschaften von Erzeugern dieses Wurzelsystems ausmachen.

\begin{defn}
  Eine Teilmenge $\Delta$ von $\Phi$ heißt \emph{Fundamentalsystem}, falls die folgenden Bedingungen erfüllt sind:
  \begin{enumerate}[(B1)]
    \item Es ist $\Delta$ eine Vektorraumbasis von $E$.
    \item Jede Wurzel $\beta \in \Phi$ lässt sich schreiben als $\beta = \sum_{\alpha \in \Delta} k_\alpha \alpha$ mit ganzzahligen Linearfaktoren $k_\alpha$ die alle dasselbe Vorzeichen besitzen.
  \end{enumerate}
  Die Elemente von $\Delta$ bezeichnet man auch als $\emph{einfache}$ Wurzeln.
\end{defn}

\begin{bem}
  Einen Beweis dafür, dass jedes Wurzelsystem eine Fundamentalsystem besitzt, findet man zum Beispiel in \cite[S.48]{humphreys1972introduction} oder \cite[S.116]{erdmann2006introduction}.
  Aus Eigenschaft (B1) von Fundamentalsystemen folgt sofort, dass die Linearfaktoren in (B2) eindeutig bestimmt sind. 
  Es lässt sich daher die Höhenfunktion 
  \begin{displaymath}
    \hgt(\beta) := \sum_{\alpha \in \Delta} k_\alpha 
  \end{displaymath}
  definieren.
  Entsprechend des Vorzeichens der Höhenfunktion bezeichnet man Wurzeln auch als \emph{positiv} oder \emph{negativ}.
  Die Menge der positiven Wurzeln bezeichnen wir im Folgenden auch mit $\Phi^+$, die der negativen Wurzeln entsprechend mit $\Phi^-$.
  Einfache Wurzeln sind stets positiv.
  Zudem induziert jedes Fundamentalsystem $\Delta$ eine Halbordnung auf $\Phi$ durch
  \begin{displaymath}
    \beta \preceq \alpha \quad \text{gilt genau dann, wenn} \quad \alpha - \beta \text{ positiv ist oder } \alpha = \beta \text{ gilt}.
  \end{displaymath}
\end{bem}

Bezüglich eines Wurzelsystems $\Phi$ lassen sich auch Vektoren des umfassenden Vektorraums klassifizieren.

\begin{defn}
  Sei $\Phi$ ein Wurzelsystem in $E$.
  Ein Vektor $\gamma \in E$ heißt \emph{regulär}, falls $\gamma \in E \setminus \bigcup_{\alpha \in \Phi} P_\alpha$.
  Die Familie $(P_\alpha)_{\alpha \in \Phi}$ liefert eine Partition von $E$ in maximal zusammenhängende Mengen, die sogenannten \emph{\weyl\hyp{}Kammern}.
  Die jedem $\gamma \in E$ eindeutig zugeordnete \weyl\hyp{}Kammer zuordnen, werde mit $\mathfrak{C}(\gamma)$ bezeichnet.
\end{defn}

Liegen zwei reguläre Wurzeln $\gamma, \gamma'$ in derselben \weyl\hyp{}Kammer, so liegen sie bezüglich allen Hyperebenen $P_\alpha$ derselben Seite, was bedeutet, dass $\sign (\gamma, \alpha) = \sign( \gamma', \alpha)$ gilt, für alle $\alpha \in \Phi$.
Bezeichnet man mit
\begin{displaymath}
  \Phi^+(\gamma) := \{ \alpha \in \Phi \mid (\gamma, \alpha) > 0 \}
\end{displaymath}
die Menge aller Wurzeln, die mit $\gamma$ einen spitzen Winkel einschließen, so gilt in diesem Falle $\Phi^+(\gamma) = \Phi^+(\gamma')$.
Bezeichnet man zudem mit $\Delta(\gamma)$ das Fundamentalsystem aller Wurzeln $\alpha \in \Phi^+(\gamma)$, die sich als Summe $\alpha = \beta_1 + \beta_2$ zweier positiver Wurzeln $\beta_1, \beta_2 \in \Phi^+(\gamma)$ schreiben lassen, so gilt zusätzlich $\Delta(\gamma) = \Delta(\gamma')$. 
Wurzeln, die sich in der oben genannten Weise ausdrücken lassen, bezeichnet man auch als \emph{zerlegbar}.

Das folgende Lemma fasst nochmals die vorangehenden Überlegungen zusammen.

\begin{lem}
  Seien $\gamma, \gamma' \in E$ regulär bezüglich des Wurzelsystems $\Phi$.
  Dann folgt aus $\mathfrak{C}(\gamma) = \mathfrak{C}(\gamma')$, dass $\Phi^+(\gamma) = \Phi^+(\gamma')$. 
  Dies ist wiederum genau dann der Fall, wenn $\Delta(\gamma) = \Delta(\gamma')$ gilt.
  Jeder \weyl\hyp{}Kammer $\mathfrak{C}(\gamma)$ entspricht also genau ein Fundamentalsystem $\Delta(\gamma)$. \qed
\end{lem}

Dass $\Delta(\gamma)$ tatsächlich ein Fundamentalsystem von $\Phi$ ist, lässt sich in \cite[S.48]{humphreys1972introduction} nachlesen.
Es gilt nämlich der folgende Satz.

\begin{thm}
  Sei $\Phi$ ein Wurzelsystem und $\gamma \in E$ diesbezüglich regulär.
  Dann ist die Menge $\Delta(\gamma)$ aller unzerlegbaren Wurzeln aus $\Phi^+(\gamma)$ ein Fundamentalsystem von $\Phi$ und jedes Fundamentalsystem ist von dieser Form. \qed
\end{thm}

Die soeben eingeführten Begriffe veranschaulicht Abbildung \ref{fig:fundamentalWeylChamber}.

\begin{figure}
  \caption{Das Wurzelsystem A2}
  \label{fig:fundamentalWeylChamber}
\end{figure}

\begin{defn}
  Sei $\Phi$ ein Wurzelsystem in $E$ mit Fundamentalsystem $\Delta$.
  Gilt für ein reguläres $\gamma \in E$, dass $\Delta = \Delta(\gamma)$, so bezeichnet man $\mathfrak{C}(\Delta) := \mathfrak{C}(\gamma)$ als \emph{Fundamentalkammer bezüglich} $\Delta$.
\end{defn}

Wir betrachten nun einen Spezialfall von Spiegelungsgruppen. Allgemeine Spiegelungsgruppen werden in \cite{humphreys1992reflection} behandelt.

\begin{defn}
  Sei $\Phi$ ein Wurzelsystem in $E$. 
  Dann bezeichnet $\W$ die von den Spiegelungen $\sigma_\alpha$, $\alpha \in \Phi$, erzeugte Untergruppe der allgemeinen linearen Gruppe $\GL(E)$. 
  Man nennt $\W$ die \emph{\weyl\hyp{}Gruppe} von $\Phi$.
\end{defn}

Es lassen sich nun unterschiedliche von $\W$ induzierte Gruppenoperationen betrachten. Über deren Wohldefiniertheit gibt die nachfolgende Proposition Auskunft. 

Für den Beweis benötigen wir ein Lemma, welches das Verhalten von Spiegelungen aus der \weyl\hyp{}Gruppe unter Konjugation mit Vektorraumautomorphismen beschreibt. 
Der Beweis dieses Lemmas findet sich in \cite[S.43]{humphreys1972introduction}.

\begin{lem}
  \label{lem:conjReflection}
  Sei $\Phi$ ein Wurzelsystem in $E$ mit \weyl\hyp{}Gruppe $\W$ und $\sigma \in \W$.
  Dann gilt $\sigma \sigma_\alpha \sigma^{-1} = \sigma_{\sigma(\alpha)}$ für alle $\alpha \in \Phi$. \qed
\end{lem}

Mit weiteren Eigenschaften dieser Gruppenoperationen werden wir uns in Abschnitt \ref{sec:weylgroup} beschäftigen.

\begin{prop}
  Sei $\Phi$ ein Wurzelsystem über $E$ mit \weyl\hyp{}Gruppe $\W$.
  Dann gelten für alle $\gamma, \gamma' \in E$ und $\sigma \in \W$ die folgenden Aussagen:
  \begin{enumerate}[(1)]
    \item $\W$ operiert auf der Menge der regulären Elemente: 
      \begin{addmargin}[2em]{2em}
        Es ist $\sigma(\gamma)$ genau dann regulär, wenn $\gamma$ regulär ist.
      \end{addmargin}
      
    \item Aus $\mathfrak{C}(\gamma) = \mathfrak{C}(\gamma')$ folgt, dass auch $\mathfrak{C}(\sigma(\gamma)) = \mathfrak{C}(\sigma(\gamma'))$. 
      Es operiert $\W$ auf der Menge der \weyl\hyp{}Kammern $\{\mathfrak{C}(\gamma) \mid \gamma \text{ regulär}\}$ durch $\mathfrak{C}(\sigma(\gamma)) = \sigma(\mathfrak{C}(\gamma))$.

    \item $\W$ operiert auf der Menge der Fundamentalsysteme $\{\Delta(\gamma) \mid \gamma \text{ regulär}\}$: 
      \begin{addmargin}[2em]{2em}
        Ist $\Delta$ ein Fundamentalsystem für $\Phi$, so auch $\sigma(\Delta)$. 
      \end{addmargin}

    \item Die unter (3) und (4) beschriebenen Gruppenoperationen sind kompatibel in dem Sinne, dass 
      $\sigma(\Delta(\gamma)) = \Delta(\sigma(\gamma))$.
  \end{enumerate}
\end{prop}

\begin{proof}
  (1): 
  Angenommen $\sigma(\gamma)$ sei nicht regulär, dann existiert ein $\alpha \in \Phi$, sodass $\sigma(\gamma) \in P_\alpha$.
  Damit folgt $\sigma(\gamma) = \sigma_\alpha \sigma(\gamma)$, da $\sigma_\alpha$ Elemente aus $P_\alpha$ fixiert.
  Hieraus folgt mit Lemma \ref{lem:conjReflection} nun $\gamma = \sigma^{-1} \sigma_\alpha \sigma(\gamma) = \sigma_{\sigma(\alpha)}(\gamma)$, was wiederum $\gamma \in P_{\sigma(\alpha)}$ impliziert.
  Zu Beginn wurde jedoch $\gamma$ als regulär vorausgesetzt.
  Es muss also auch $\sigma(\gamma)$ regulär sein.
  
  Die Umkehrung der Aussage folgt aus der gezeigten Implikationsrichtung durch Anwendung der Spiegelung $\sigma^{-1}$ auf $\sigma(\gamma)$.

  (2): 
  Es gelte $\mathfrak{C}(\gamma) = \mathfrak{C}(\gamma')$.
  Angenommen $\mathfrak{C}(\sigma(\gamma)) \neq \mathfrak{C}(\sigma(\gamma'))$.
  Dann existiert ein $\alpha \in \Phi$, sodass $(\sigma(\gamma), \alpha) > 0$ während andererseits $(\sigma(\gamma'), \alpha) < 0$.
  Wir betrachten nun die Abbildung $x \mapsto  (\sigma(x), \alpha)$.
  Diese ist als Linearform stetig bezüglich der \textsc{euklid}ischen Topologie auf $E$.
  Da $\mathfrak{C}(\gamma)$ eine zusammenhängende Menge ist, folgt mit dem Zwischenwertsatz \cite[S.232]{bartsch2015allgemeine}, dass ein reguläres $x \in \mathfrak{C}(\gamma)$ existiert mit $(\sigma(x), \alpha) = 0$.
  Dies bedeutet jedoch gerade, dass $\sigma(x) \in P_\alpha$ und damit wäre $\sigma(x)$ nicht regulär im Widerspruch zu (1). Es muss also auch $\mathfrak{C}(\sigma(\gamma)) = \mathfrak{C}(\sigma(\gamma'))$ gelten.
  
  Sei nun für den zweiten Teil der zu beweisenden Aussage $x \in \mathfrak{C}(\sigma(\gamma))$.
  Daraus folgt $\mathfrak{C}(x) = \mathfrak{C}(\sigma(\gamma))$.
  Dann gilt nach der soeben bewiesenen Aussage auch $\mathfrak{C}(\sigma^{-1}(x)) = \mathfrak{C}(\gamma)$.
  Insbesondere gilt also $\sigma^{-1}(x) \in \mathfrak{C}(\gamma)$ und damit auch $x \in \sigma(\mathfrak{C}(\gamma))$
  
  Die umgekehrte Ungleichung folgt analog: Ist $x \in \sigma(\mathfrak{C}(\gamma))$, so ist $\sigma(x) \in \mathfrak{C}(\gamma)$, also $\mathfrak{C}(\sigma(x)) = \mathfrak{C}(\gamma)$. 
  Wie schon gezeigt, folgt daraus unter Betrachtung von Bildern unter $\sigma^{-1} = \sigma$ sofort $\mathfrak{C}(x) = \mathfrak{C}(\sigma(\gamma))$, also insbesondere $x \in \mathfrak{C}(\sigma(\gamma))$.

  (3): 
  Da $\sigma$ als orthogonale Abbildung insbesondere injektiv ist, folgt, dass $\sigma(\Delta)$ wieder ein System linear unabhängiger Vektoren und damit aufgrund von (B1) wieder eine Basis von $E$ ist.
  
  Ist nun $\beta \in \Phi$ gegeben, so gilt
  $\sigma(\beta) 
    = \sigma(\sum_{\alpha \in \Delta} k_\alpha \alpha)
    = \sum_{\alpha \in \Delta} k_\alpha \sigma(\alpha)
    = \sum_{\sigma(\alpha) \in \sigma(\Delta)} k_\alpha \sigma(\alpha)
  $
  aufgrund der Linearität von $\sigma$.
  Die Vorzeichen der Linearfaktoren $k_\alpha$ bleiben zudem unverändert, womit (B2) folgt.
  Damit ist auch $\sigma(\Delta)$ ein Fundamentalsystem.

  (4):
  Wir stellen zunächst fest, dass aufgrund der Aussage (3), die zu zeigende Gleichheit in dem Sinne wohldefiniert ist, dass $\sigma(\Delta(\gamma))$ als Bild eines Fundamentalsystems wieder ein Fundamentalsystem darstellt.

  Sei nun $\sigma(\alpha) \in \sigma(\Delta(\gamma))$.
  Da $\alpha \in \Delta(\gamma)$, gilt definitionsgemäß $(\alpha, \gamma) > 0$.
  Spiegelungen erhalten als Isometrien das Skalarprodukt, es gilt also auch $(\sigma(\alpha), \sigma(\gamma)) > 0$.
  Dies wiederum impliziert $\sigma(\alpha) \in \Phi^+(\sigma(\gamma))$.
  Angenommen $\sigma(\alpha)$ wäre eine zerlegbare Wurzel.
  Dann ist auch $\alpha$ zerlegbar, da $\sigma$ bijektiv ist.
  Dies steht jedoch im Wiederspruch dazu, dass mit $\alpha \in \Delta$ die Wurzel $\alpha$ als einfach vorausgesetzt wurde.
  Also muss $\sigma(\alpha)$ auch einfach sein, womit $\sigma(\alpha) \in \sigma(\Delta(\gamma))$ folgt.
  Das bedeutet, dass $\sigma(\Delta(\gamma)) \subseteq \Delta(\sigma(\gamma))$.
  
  Da $\sigma$ bijektiv ist und beide Mengen als Teilmengen des endlichen Wurzelsystems $\Phi$ auch endlich sind, folgt hiermit bereits die Gleichheit.
\end{proof}
