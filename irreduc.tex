\section{Irreduzible Wurzelsysteme}

In diesem Abschnitt beschäftigen wir uns mit den kleinsten ``Bausteinen'' aus denen Wurzelsysteme zusammengesetzt sind.

\begin{defn}
  Ein Wurzelsystem $\Phi$ in einem \textsc{euklid}ischen Vektorraum $E$ heißt \emph{irreduzibel} wenn keine Zerlegung von $\Phi$ in zwei disjunkte Mengen existiert, sodass die Elemente aus unterschiedlichen Mengen paarweise orthogonal sind.
  Ebenso bezeichne man ein Fundamentalsystem eines Wurzelsystems als irreduzibel, falls keine disjunkte Zerlegung im obigen Sinne existiert.
\end{defn}

\begin{lem}
  Sei $\Delta$ ein Fundamentalsystem eines Wurzelsystems $\Phi$ in $E$.
  Dann ist $\Phi$ genau dann irreduzibel, wenn $\Delta$ irreduzibel ist.
\end{lem}

\begin{proof}
  ``$\Rightarrow$'':
  Angenommen, es existiere eine Zerlegung $\Phi = \Phi_1 \cup \Phi_2$ mit $(\Phi_1, \Phi_2)$ = 0.
  Falls $\Delta \not\subseteq \Phi_i$ gilt für ein $i \in \{1,2\}$, so ist $\Delta = (\Phi_1 \cap \Delta) \cup (\Phi_2 \cap \Delta)$ eine entsprechende Zerlegung des Fundamentalsystems.
  Anderfalls ist ohne Beschränkung der Allgemeinheit $\Delta \subseteq \Phi_1$ und damit $(\Delta, \Phi_2) = 0$
  Da nach $(B1)$ das Fundamentalsystem eine Vektorraumbasis ist, folgt mit der Linearität des Skalarproduktes sogleich $(E, \Phi_2) = 0$ und da Skalarprodukte nicht ausgeartet sind letztlich $\Phi_2 = {0}$ im Widerspruch zu $(B1)$.

  ``$\Leftarrow$'':
  Angenommen, $\Phi$ sei irreduzibel aber es existiere eine Zerlegung $\Delta = \Delta_1 \cup \Delta_2$ mit $(\Delta_1, \Delta_2) = 0$.
  Wir definieren für $i \in \{1,2\}$ die Mengen $\Phi_i$ aller Wurzeln, die in der $\W$\hyp{}Bahn einer einfachen Wurzel liegen. 
  Nach Satz \ref{thm:simplyTransitive}(3) gilt $\Phi = \Phi_1 \cup \Phi_2$.

  Sei $\gamma$ eine Wurzel in $\Phi_i$.
  Nach \ref{thm:simplyTransitive}(3) existiert ein $\sigma \in \W$, sodass $\sigma(\alpha) = \gamma$ für ein $\alpha \in \Delta$.
  Es existiert eine Darstellung $\sigma = \sigma_{\alpha_1} \dots \sigma_{\alpha_t}$ ($\ast$) mit $\alpha_i \in \Delta$.
  Wir können ohne Beschränkung der Allgemeinheit $\alpha_i \in \Delta_1$ für alle $i$ annehmen.
  Denn da nach Voraussetzung $(\Delta_1, \Delta_2) = 0$ gilt, folgt für die Darstellung ($\ast$), dass
  \begin{displaymath}
    \sigma = \sigma_{\alpha_1} \dots \sigma_{\alpha_{t-j}} \sigma_{\alpha_{t-j+1}} \dots \sigma_{\alpha_t}
  \end{displaymath}
  mit $\alpha_k \in \Delta_2$ für $k \geq t-j+1$ gilt.
  Da $\alpha$ aus $\Delta_1$ stammt und für $\alpha_k \in \Delta_2$ gilt, dass $\sigma_{\alpha_j}(\alpha) = \alpha - \langle \alpha, \alpha_j \rangle \alpha_j = \alpha$ folgt sogleich $\sigma = \sigma_{\alpha_1} \dots \sigma_{\alpha_{t - j}}$. 
  Damit folgt schließlich
  $\gamma = \sigma(\alpha) = \sigma_{\alpha_1} \dots \sigma_{\alpha_k}(\alpha)$.

  Daher liegt $\Phi_i$ im Teilraum $E_i \subseteq E$, welcher von $\Delta_i$ aufgespannt wird.
  Mit der Bilinearität des Skalaproduktes folgt sodann $(\Phi_1, \Phi_2) = 0$.
  Da Skalarprodukte nicht ausgeartet sind folgt $\Phi_1 = \emptyset$ oder $\Phi_2 = \emptyset$, was wiederum $\Delta_1 = \emptyset$ beziehungsweise $\Delta-2 = \emptyset$ impliziert, im Widerspruch zur Voraussetzung.
\end{proof}
