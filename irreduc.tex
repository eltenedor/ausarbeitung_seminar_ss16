\section{Irreduzible Wurzelsysteme}
\label{sec:irreduc}

In diesem Abschnitt beschäftigen wir uns mit den Eigenschaften der kleinsten ``Bausteine'' aus denen Wurzelsysteme zusammengesetzt sind.
Es sei wieder $\Phi$ ein fest gewähltes Wurzelsystem über dem \euklid ischen Vektorraum $E$ mit Fundamentalsystem $\Delta$ und zugehöriger \weyl\hyp{}Gruppe $\W$.

\begin{defn}
  Ein Wurzelsystem $\Phi$ heißt \emph{irreduzibel}, wenn keine Zerlegung von $\Phi$ in zwei nichtleere disjunkte Mengen existiert, sodass die Elemente aus unterschiedlichen Mengen paarweise orthogonal sind.
  Ebenso bezeichne man ein Fundamentalsystem eines Wurzelsystems als \emph{irreduzibel}, falls keine disjunkte Zerlegung im obigen Sinne existiert.
\end{defn}

\begin{lem}
  \label{lem:irreducibleRoot}
  Das Wurzelsystem $\Phi$ ist genau dann irreduzibel, wenn $\Delta$ irreduzibel ist.
\end{lem}

\begin{proof}
  ``$\Rightarrow$'':
  Angenommen, es existiere eine Zerlegung $\Phi = \Phi_1 \cup \Phi_2$ mit $(\Phi_1, \Phi_2)$ = 0.
  Falls für ein $i \in \{1,2\}$ die Relation $\Delta \not\subseteq \Phi_i$ gilt, ist $\Delta = (\Phi_1 \cap \Delta) \cup (\Phi_2 \cap \Delta)$ eine entsprechende Zerlegung des Fundamentalsystems.
  Andernfalls ist ohne Beschränkung der Allgemeinheit $\Delta \subseteq \Phi_1$ und damit $(\Delta, \Phi_2) = 0$.
  Da nach \hyperref[it:B1]{(B1)} das Fundamentalsystem eine Vektorraumbasis ist, folgt mit der Linearität des Skalarproduktes sogleich $(E, \Phi_2) = 0$ und da Skalarprodukte nicht ausgeartet sind letztlich $\Phi_2 = {0}$ im Widerspruch zu \hyperref[it:B1](B1).

  ``$\Leftarrow$'':
  Angenommen, $\Phi$ sei irreduzibel und es existiere eine Zerlegung $\Delta = \Delta_1 \cup \Delta_2$ in disjunkte nichtleere Teilmengen $\Delta_1$ und $\Delta_2$ mit $(\Delta_1, \Delta_2) = 0$.
  Wir definieren für alle $i \in \{1,2\}$ die Mengen $\Phi_i$ aller Wurzeln, die in der $\W$\hyp{}Bahn einer Wurzel aus $\Delta_i$ liegen. 
  Nach Satz \ref{thm:simplyTransitive}(c) liegt jede Wurzel in der $\W$\hyp{}Bahn einer einfachen Wurzel, womit $\Phi = \Phi_1 \cup \Phi_2$ gilt.

  Sei $\gamma$ eine Wurzel in $\Phi$.
  Wegen Satz \ref{thm:simplyTransitive}(c) existieren ein $\sigma \in \W$ und ein $\alpha \in \Delta$, sodass $\sigma(\alpha) = \gamma$ gilt.
  Aus Symmetriegründen nehmen wir an, dass $\alpha$ in $\Delta_1$ und somit $\gamma$ in $\Phi_1$ liegt.

  Es existiert eine Darstellung $\sigma = \sigma_{\alpha_1} \dots \sigma_{\alpha_t}$ ($\ast$) mit $\alpha_i \in \Delta$.
  Wir können ohne Beschränkung der Allgemeinheit $\alpha_1,\dots,\alpha_t \in \Delta_1$ annehmen.
  Denn, da nach Voraussetzung $(\Delta_1, \Delta_2) = 0$ gilt, folgt mit Lemma \ref{lem:orthogonalRoots} für die Darstellung ($\ast$), dass sich die Faktoren umordnen lassen zu
  \begin{displaymath}
    \sigma = \sigma_{\alpha_1} \dots \sigma_{\alpha_j} \sigma_{\alpha_{j+1}} \dots \sigma_{\alpha_t},
  \end{displaymath}
  wobei $1 \leq j < t$ und $\alpha_k \in \Delta_2$ für alle $k > j$ gilt.
  Da $\alpha$ aus $\Delta_1$ stammt und für $\alpha_k \in \Delta_2$ die Identität $\sigma_{\alpha_k}(\alpha) = \alpha - \langle \alpha, \alpha_k \rangle \alpha_k = \alpha$ gilt, folgt sogleich $\sigma = \sigma_{\alpha_1} \dots \sigma_{\alpha_j}$. 
  Damit ergibt sich schließlich $\gamma = \sigma(\alpha) = \sigma_{\alpha_1} \dots \sigma_{\alpha_j}(\alpha)$.

  Aus der Darstellung der Spiegelung an einer einfachen Wurzel folgt, dass $\gamma$ in $E_1 := \spn(\Delta_1)$ liegt.
  Es gilt also auch $\Phi_1 \subseteq E_1$.
  Mit der Bilinearität des Skalarproduktes folgt dann $(\Phi_1, \Phi_2) = 0$
  Da das Skalarprodukt nicht ausgeartet ist, muss dann $\Phi_1$ oder $\Phi_2$ leer sein.
  Dies impliziert jedoch, dass im Widerspruch zur Voraussetzung $\Delta_1$ oder $\Delta_2$ leer ist.
\end{proof}

Das nächste Lemma beschäftigt sich mit bezüglich der Halbordnung $\succeq$ auf $\Phi$ maximalen Wurzeln. 
Ihre Existenz ist aufgrund der in \hyperref[it:R1]{(R1)} vorausgesetzten Endlichkeit von $\Phi$ immer gewährleistet. 

\begin{lem}
  \label{lem:maximalRoot}
  Sei $\Phi$ ein irreduzibles Wurzelsystem.
  Bezüglich der Halbordnung $\succeq$ auf $\Phi$ existiert genau eine maximale Wurzel $\beta$.
  Insbesondere folgt aus $\alpha \neq \beta$ auch $\hgt \alpha < \hgt \beta$ und $(\beta, \alpha) \geq 0$ für alle $\alpha \in \Delta$.
  In der Darstellung $\beta = \sum_{\alpha \in \Delta} k_\alpha \alpha$ sind alle Koeffizienten $k_\alpha$ strikt positiv.
\end{lem}

\begin{proof}
  Sei $\beta = \sum_{\alpha \in \Delta} k_\alpha \alpha$ eine maximale Wurzel bezüglich der Relation $\succeq$.
  Da jede einfache Wurzel bereits positiv ist, gilt zunächst auch $\beta \succeq 0$.
  Wir definieren $\Delta_1 = \{\alpha \in \Delta \mid k_\alpha > 0\}$ und $\Delta_2 = \{\alpha \in \Delta \mid k_\alpha = 0\}$.
  Aufgrund der Positivität von $\beta$ ist damit $\Delta = \Delta_1 \cup \Delta_2$ eine disjunkte Vereinigung.

  Dass $\Delta_1$ nicht leer ist, folgt bereits aus der Positivität von $\beta$.
  Wir wollen nun zeigen, dass $\Delta_2$ leer ist.
  Angenommen, $\Delta_2$ sei nicht leer.
  Es gilt dann $(\alpha', \alpha) \leq 0$ für alle $\alpha' \in \Delta_2$ und $\alpha \in \Delta$, denn sonst wäre nach Lemma \ref{lem:sumDiffRoot} auch $\alpha' - \alpha$ eine Wurzel, was jedoch \hyperref[it:B2]{(B2)} widerspricht.
  Damit folgt sogleich $(\alpha', \beta) = \sum_{\alpha \in \Delta} k_\alpha (\alpha', \alpha) \leq 0 $ ($\ast$).

  Da $\Phi$ nach Voraussetzung irreduzibel ist, muss ein $\alpha' \in \Delta_2$ existieren, welches nicht orthogonal zu $\Delta_1$ ist und für das $(\alpha', \alpha) < 0$ gilt.
  Ansonsten folgt nämlich aus der Zerlegung von $\Delta$ in $\Delta_1$ und $\Delta_2$ die Reduzibilität von $\Delta$, also nach Lemma \ref{lem:irreducibleRoot} auch die Reduzibilität von $\Phi$, was jedoch der Voraussetzung widerspricht.

  Seien also $\alpha \in \Delta_1$ und $\alpha' \in \Delta_2$ mit $(\alpha, \alpha') < 0$ gegeben.
  Damit folgt dann, dass in ($\ast$) sogar die strikte Ungleichung $(\alpha', \beta) < 0$ gilt.
  Lemma \ref{lem:sumDiffRoot} ergibt, dass $\alpha' +  \beta$ eine positive Wurzel ist.
  Dies widerspricht aber der vorausgesetzten Maximalität von $\beta$
  Die anfängliche Annahme $\Delta_2 \neq \emptyset$ muss also verworfen werden.
  Also sind alle Linearfaktoren $k_\alpha$ strikt positiv.

  Zudem zeigt der vorangehende Beweis, dass $(\alpha', \beta) \geq 0$ gilt für alle $\alpha' \in \Delta$ und dass zumindest ein $\alpha' \in \Delta$ existiert mit $(\alpha', \beta) > 0$.
    Denn $\Delta$ ist nach \hyperref[it:B2]{(B2)} eine Vektorraumbasis für $E$ und falls für alle $\alpha' \in \Delta$ die Gleichheit $(\alpha', \beta) = 0$ gilt, folgt $(v, \beta) = 0$ für alle $v \in E$ und damit $\beta = 0$, da Skalarprodukte nicht ausgeartet sind.

  Nun wollen wir die Eindeutigkeit der maximalen Wurzel zeigen.
  Sei dazu $\beta'$ eine weitere Wurzel.
  Nach \hyperref[it:B2]{(B2)} und den vorangehenden Ausführungen existiert wieder eine Darstellung der Form $\beta' = \sum_{\alpha \in \Delta} k_\alpha' \alpha$ mit $k_\alpha' > 0$ für alle $\alpha \in \Delta$.
  Ebenso existiert mindestens ein $\alpha' \in \Delta$ mit $(\alpha', \beta') > 0$.
  Damit folgt nun
  \begin{displaymath}
    (\beta, \beta') = \sum_{\alpha \in \Delta} k_\alpha (\alpha, \beta') \geq k_{\alpha'} (\alpha', \beta') > 0.
  \end{displaymath}
  Mit Lemma \ref{lem:sumDiffRoot} ergibt sich nun, dass auch $\beta - \beta'$ eine Wurzel ist, es sei denn, dass $\beta$ und $\beta'$ proportional sind. 
  In diesem Fall gilt $\beta = \beta'$, da beide Wurzeln voraussetzungsgemäß positiv sind.
  Angenommen, es gelten $\beta \neq \beta'$ und $\beta - \beta' \succeq 0$.
  Dann gilt auch $\beta \succeq \beta'$ im Widerspruch zur Maximalität von $\beta$.
  Analog zeigt man, dass auch die Annahme $\beta' \succeq \beta$ verworfen werden muss.
  Also gilt $\beta = \beta'$ und die maximale Wurzel ist eindeutig.
\end{proof}

\begin{lem}
  \label{lem:irreducibleGroupOp}
  Sei $\Phi$ ein irreduzibles Wurzelsystem.
  Dann ist auch die Gruppenoperation von $\W$ irreduzibel über $E$, es existieren also außer $\{0\}$ und $E$ keine weiteren $\W$\hyp{}invarianten Unterräume von $E$.
  Insbesondere erzeugt die $\W$\hyp{}Bahn einer beliebigen Wurzel $\alpha$ den gesamten Vektorraum $E$.
\end{lem}

\begin{proof}
  Sei $E'$ ein nichttrivialer $\W$\hyp{}invarianter Untervektorraum von $E$.
  Dann ist auch das orthogonale Komplement $(E')^\perp$ ein $\W$\hyp{}invarianter Unterraum.
  Ist nämlich $\alpha \in E'$ und $\beta \in (E')^\perp$, so gilt für $\sigma \in \W$ die Gleichung
  $\langle \alpha, \sigma(\beta) \rangle = \langle \sigma^{-1}(\alpha), \beta \rangle = 0$, da $\sigma^{-1}(\alpha) \in E'$ gilt.
  Es folgt also auch $\sigma(\beta) \in (E')^\perp$ und damit die $\W$\hyp{}Invarianz des Untervektorraums $(E')^\perp$. 
  Zudem gilt die Identität $E = E' \oplus (E')^\perp$.

  Dann ist aber auch $\Phi = (\Phi \cap E') \cup (\Phi \cap (E')^\perp$ eine Zerlegung von $\Phi$ in zueinander orthogonale Teilmengen.
  Da aber $\Phi$ als irreduzibles Wurzelsystem vorausgesetzt wurde muss eine der beteiligten Teilmengen leer sein.
  Angenommen, es gelte $\Phi \subseteq E'$. 
  Da Wurzelsysteme nach \hyperref[it:W1]{(W1)} den Vektorraum erzeugen, gilt $E = E'$. 
  Im Fall $\Phi \subseteq (E')^\perp$ folgt $E = (E')^\perp$.
  Damit haben wir den ersten Teil der Behauptung bewiesen.

  Es ist der Spann einer $\W$\hyp{}Bahn einer Wurzel $\alpha$ ein nicht trivialer $\W$\hyp{}invarianter Untervektorraum von $E$.
  Mit dem vorangehenden Beweis muss $\spn(\W\alpha) = E$ gelten.
  Damit ist die Behauptung vollständig bewiesen.
\end{proof}

\begin{bem}
  Der Beweis zeigt bereits auf, wie sich reduzible Wurzelsysteme aus irreduziblen Wurzelsystemen zusammensetzen.
  Zu jeder disjunkten Zerlegung des Wurzelsystems korrespondiert nämlich eine orthogonale Summe $\W$\hyp{}invarianter Untervektorräume.
  Eine Weiterführung dieses Gedankens zur Klassifikation von Wurzelsystemen findet sich in \cite[S.57ff.]{humphreys1972introduction}.
\end{bem}

\begin{lem}
  \label{lem:rootLength}
  Sei $\Phi$ irreduzibel.
  Es bezeichne $||\alpha|| := \sqrt{(\alpha,\alpha)}$ die vom Skalarprodukt auf $E$ induzierte Norm.
  Dann nimmt die Norm auf $\Phi$ höchstens zwei unterschiedliche Werte an.
  Zudem liegen alle Wurzeln mit gleicher Norm in einer gemeinsamen $\W$\hyp{}Bahn.
\end{lem}

\begin{proof}
  Seien $\alpha, \beta \in \Phi$.
  Dann können nicht alle Elemente aus der Bahn $\W\alpha$ orthogonal zu $\beta$ sein, da nach Lemma \ref{lem:irreducibleGroupOp} die Bahn $\W\alpha$ den Vektorraum $E$ erzeugt und $\beta$ sonst $0$ sein müsste.
  Aus \cite[S.45]{humphreys1972introduction} ist bekannt, dass  
  \begin{displaymath}
    \frac{(\beta,\beta)}{(\alpha,\alpha)} \in \left\{1,2,3,\frac{1}{2},\frac{1}{3}\right\} \tag{$\ast$}
  \end{displaymath}
  gilt, vorausgesetzt es ist $(\alpha, \beta) \neq 0$.
  Daraus folgt bereits der erste Teil der Behauptung.
  Denn, angenommen es existieren drei unterschiedliche Wurzellängen, dann müssten in ($\ast$) auch die Verhältnisse $\tfrac{2}{3}$ und $\tfrac{3}{2}$ auftauchen, was jedoch nicht der Fall ist.

  Seien nun $\alpha, \beta$ Wurzeln gleicher Norm.
  Da nach Lemma \ref{lem:irreducibleGroupOp} die Bahn $\W\alpha$ den Vektorraum $E$ erzeugt, ist $\beta$ nicht orthogonal zu allen Bahnelementen.
  Wir können daher ohne Beschränkung der Allgemeinheit wir $(\alpha, \beta) \neq 0$ annehmen.
  Ansonsten ersetzen wir $\alpha$ durch ein $\W$\hyp{}Bahnelement $\sigma(\alpha)$ mit $(\sigma(\alpha), \beta) \neq 0$.
  Im Falle $\alpha = \beta$ gibt es nichts zu zeigen, daher nehmen wir an, es gelte $\alpha \neq \beta$.
  Nun folgt unter Berücksichtigung von \cite[S.45]{humphreys1972introduction}, dass $\langle \alpha, \beta \rangle = \langle \beta, \alpha \rangle = \pm 1$ gelten muss.
  Aufgrund der im vorigen Ausdruck für $\alpha$ und $\beta$ geltenden Symmetrie ist der im Allgemeinen nur im ersten Argument lineare Term $\langle \alpha, \beta \rangle$ nun sogar bilinear.
  Wir können damit, falls nötig, $\beta$ auch durch $-\beta = \sigma_\beta(\beta)$ ersetzen, da diese Elemente in einer gemeinsamen $\W$\hyp{}Bahn liegen.
  Daher lässt sich annehmen, dass $\langle \alpha, \beta \rangle = 1$ gilt.
  Damit folgt
  \begin{align*}
    \sigma_\alpha \sigma_\beta \sigma_\alpha(\beta) 
    &= \sigma_\alpha \sigma_\beta(\beta - \langle \beta, \alpha \rangle \alpha) \\
    &= \sigma_\alpha \sigma_\beta(\beta - \alpha) \\
    &= \sigma_\alpha (-\beta - \alpha + \langle \alpha, \beta \rangle \beta) \\
    &= \sigma_\alpha (-\beta - \alpha + \beta) \\
    &= \sigma_\alpha (-\alpha) \\
    &= \alpha
  \end{align*}
  und es liegen $\alpha$ und $\beta$ in der selben $\W$\hyp{}Bahn.
\end{proof} 

\begin{bem}
  Ist das Wurzelsystem $\Phi$ irreduzibel und existieren Wurzeln unterschiedlicher Norm, so ist es auf Grundlage von Lemma \ref{lem:rootLength} möglich von \emph{langen} und \emph{kurzen} Wurzeln zu sprechen.
  Haben alle Wurzeln aus $\Phi$ dieselbe Norm, so ist es üblich alle Wurzeln als lang zu bezeichnen.
\end{bem}

\begin{lem}
  Sei $\Phi$ ein irreduzibles Wurzelsystem mit Wurzeln unterschiedlicher Norm.
  Dann ist die im Sinne von Lemma \ref{lem:maximalRoot} maximale Wurzel $\beta$ eine lange Wurzel.
\end{lem}

\begin{proof}
  Sei $\alpha \in \Phi$ gegeben.
  Da die Normabbildung eingeschränkt auf $\Phi$ nur zwei Werte annimmt, reicht es aus $(\beta, \beta) \geq (\alpha, \alpha)$ zu zeigen.
  Dazu ersetzen wir $\alpha$ und $\beta$ durch die nach \ref{lem:fundamentalDomain} eindeutig bestimmten Wurzeln aus $\W\alpha$ beziehungsweise $\W\beta$, die im Abschluss der Fundamentalkammer bezüglich $\Delta$ liegen. 
  Dies ist ohne Einschränkungen möglich, da Lemma \ref{lem:rootLength} garantiert, dass alle Bahnelemente dieselbe Norm besitzen.
  Nach Lemma \ref{lem:maximalRoot} gilt aufgrund der Maximalität von $\beta$ nun $\beta - \alpha \succeq 0$.
  Weiter ist $(\gamma, \beta - \alpha) \geq 0$ für alle $\gamma \in \overline{\mathfrak{C}(\Delta)}$, also insbesondere auch für $\gamma = \alpha$ und $\gamma = \beta$, was zu den Ungleichungen
  \begin{displaymath}
    (\beta, \beta) - (\beta, \alpha) \geq 0 \quad\text{und}\quad 
    (\beta, \alpha) - (\alpha, \alpha) 
    = (\alpha, \beta) - (\alpha, \alpha) \geq 0,
  \end{displaymath}
  also
  \begin{displaymath}
    (\beta, \beta)  \geq (\beta, \alpha) \quad\text{und}\quad 
    (\beta, \alpha) \geq (\alpha, \alpha)
  \end{displaymath}
  führt.
  Beide Ungleichungen zusammen ergeben folglich
  \begin{displaymath}
    (\beta, \beta) \geq (\beta, \alpha) \geq (\alpha, \alpha),
  \end{displaymath}
  was die Behauptung beweist.
\end{proof}
