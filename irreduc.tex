\section{Irreduzible Wurzelsysteme}

In diesem Abschnitt beschäftigen wir uns mit den Eigenschaften der kleinsten ``Bausteine'' aus denen Wurzelsysteme zusammengesetzt sind.
Es sei wieder $\Phi$ ein festgewähltes Wurzelsystem im \textsc{euklid}ischen Vektorraum $E$ mit Fundamentalsystem $\Delta$ und zugehöriger \weyl\hyp{}Gruppe $\W$.

\begin{defn}
  Ein Wurzelsystem $\Phi$ in einem \textsc{euklid}ischen Vektorraum $E$ heißt \emph{irreduzibel} wenn keine Zerlegung von $\Phi$ in zwei nichtleere disjunkte Mengen existiert, sodass die Elemente aus unterschiedlichen Mengen paarweise orthogonal sind.
  Ebenso bezeichne man ein Fundamentalsystem eines Wurzelsystems als irreduzibel, falls keine disjunkte Zerlegung im obigen Sinne existiert.
\end{defn}

\begin{lem}
  \label{lem:irreducibleRoot}
  Sei $\Delta$ ein Fundamentalsystem eines Wurzelsystems $\Phi$ in $E$.
  Dann ist $\Phi$ genau dann irreduzibel, wenn $\Delta$ irreduzibel ist.
\end{lem}

\begin{proof}
  ``$\Rightarrow$'':
  Angenommen, es existiere eine Zerlegung $\Phi = \Phi_1 \cup \Phi_2$ mit $(\Phi_1, \Phi_2)$ = 0.
  Falls $\Delta \not\subseteq \Phi_i$ gilt für ein $i \in \{1,2\}$, so ist $\Delta = (\Phi_1 \cap \Delta) \cup (\Phi_2 \cap \Delta)$ eine entsprechende Zerlegung des Fundamentalsystems.
  Anderfalls ist ohne Beschränkung der Allgemeinheit $\Delta \subseteq \Phi_1$ und damit $(\Delta, \Phi_2) = 0$.
  Da nach \hyperref[it:B1]{(B1)} das Fundamentalsystem eine Vektorraumbasis ist, folgt mit der Linearität des Skalarproduktes sogleich $(E, \Phi_2) = 0$ und da Skalarprodukte nicht ausgeartet sind letztlich $\Phi_2 = {0}$ im Widerspruch zu \hyperref[it:B1](B1).

  ``$\Leftarrow$'':
  Angenommen, $\Phi$ sei irreduzibel aber es existiere eine Zerlegung $\Delta = \Delta_1 \cup \Delta_2$ mit $(\Delta_1, \Delta_2) = 0$.
  Wir definieren für $i \in \{1,2\}$ die Mengen $\Phi_i$ aller Wurzeln, die in der $\W$\hyp{}Bahn einer einfachen Wurzel liegen. 
  Nach Satz \ref{thm:simplyTransitive}(c) gilt $\Phi = \Phi_1 \cup \Phi_2$.

  Sei $\gamma$ eine Wurzel in $\Phi$.
  Wegen Satz \ref{thm:simplyTransitive}(c) existieren ein $\sigma \in \W$ und ein $\alpha \in \Delta$, sodass $\sigma(\alpha) = \gamma$ gilt.
  Nehmen wir zunächst an, dass $\alpha$ in $\Delta_1$ liegt.

  Es existiert eine Darstellung $\sigma = \sigma_{\alpha_1} \dots \sigma_{\alpha_t}$ ($\ast$) mit $\alpha_i \in \Delta$.
  Wir können ohne Beschränkung der Allgemeinheit $\alpha_i \in \Delta_1$ für alle $i$ annehmen.
  Denn da nach Voraussetzung $(\Delta_1, \Delta_2) = 0$ gilt, folgt mit Lemma \ref{lem:orthogonalRoots} für die Darstellung ($\ast$), dass sich die Faktoren umordnen lassen zu
  \begin{displaymath}
    \sigma = \sigma_{\alpha_1} \dots \sigma_{\alpha_j} \sigma_{\alpha_j+1} \dots \sigma_{\alpha_t},
  \end{displaymath}
  wobei $1 \leq j \leq t$ und $\alpha_k \in \Delta_2$ für alle $k > j$.
  Da $\alpha$ aus $\Delta_1$ stammt und für $\alpha_k \in \Delta_2$ die Identität $\sigma_{\alpha_k}(\alpha) = \alpha - \langle \alpha, \alpha_k \rangle \alpha_k = \alpha$ gilt, folgt sogleich $\sigma = \sigma_{\alpha_1} \dots \sigma_{\alpha_j}$. 
  Damit ergibt sich schließlich $\gamma = \sigma(\alpha) = \sigma_{\alpha_1} \dots \sigma_{\alpha_k}(\alpha)$.

  Aus der Darstellung der Spiegelung an einer einfachen Wurzel folgt, dass $\gamma$ im Spann $E_1$ von $\Delta_1$ liegt.
  Dann gilt aber $\Phi \subseteq E_1$ und folglich $E = E_1$
  Damit wäre dann aber $\Delta_2$ leer im Widerspruch zur Voraussetzung.
\end{proof}

Das nächste Lemma beschäftigt sich mit bezüglich der Halbordnung $\succeq$ auf $\Phi$ maximalen Wurzeln, deren Existenz aufgrund der in \hyperref[it:R1]{(R1)} vorausgesetzten Endlichkeit von $\Phi$ immer gewährleistet ist. 

\begin{lem}
  \label{lem:maximalRoot}
  Sei $\Phi$ ein irreduzibles Wurzelsystem in $E$ mit Fundamentalsystem $\Delta$.
  Bezüglich der Halbordnung $\succeq$ auf $\Phi$ existiert genau eine maximale Wurzel $\beta$.
  Insbesondere folgt aus $\alpha \neq \beta$ auch $\hgt \alpha < \hgt \beta$ und $(\beta, \alpha) \geq 0$ für alle $\alpha \in \Delta$.
  In der Darstellung $\beta = \sum_{\alpha \in \Delta} k_\alpha \alpha$ sind alle Koeffizienten $k_\alpha$ strikt positiv.
\end{lem}

\begin{proof}
  Sei $\beta = \sum_{\alpha \in \Delta} k_\alpha \alpha$ eine maximale Wurzel bezüglich $\succeq$.
  Da jede einfache Wurzel bereits positiv ist, gilt zunächst auch $\beta \succeq 0$.
  Wir definieren $\Delta_1 = \{\alpha \in \Delta \mid k_\alpha > 0\}$ und $\Delta_2 = \{\alpha \in \Delta \mid k_\alpha = 0\}$.
  Aufgrund der Positivität von $\beta$ ist damit $\Delta = \Delta_1 \cup \Delta_2$ eine disjunkte vereinigung.

  Wir wollen nun zeigen dass $\Delta_2$ leer ist.
  Angenommen, $\Delta_2$ sei nicht leer.
  Es gilt dann $(\alpha', \alpha) \leq 0$ für alle $\alpha' \in \Delta_2$ und $\alpha \in \Delta$, denn sonst wäre $\alpha' - \alpha$ eine Wurzel, was jedoch \hyperref[it:B2]{(B2)} widerspricht.
  Damit folgt sogleich $(\alpha', \beta) = \sum_{\alpha \in \Delta} k_\alpha (\alpha', \alpha) \leq 0 $ ($\ast$).
  Da $\Phi$ nach Voraussetzung irreduzibel ist, muss ein $\alpha' \in \Delta_2$ existieren, weches nicht orthogonal zu $\Delta_1$ ist.
  Ansonsten folgt nämlich aus der Zerlegung in $\Delta_1$ und $\Delta_2$ die Reduzibilität von $\Delta$, also nach Lemma \ref{lem:irreducibleRoot} auch die Reduzibilität von $\Phi$, was jedoch der Voraussetzung widerspricht.

  Sei also $\alpha' \in \Delta_2$ mit $(\alpha, \alpha') < 0$ gegeben.
  Damit folgt dann, dass in ($\ast$) sogar die strikte Ungleichung $(\alpha', \beta) < 0$ gilt.
  Mit Lemma \ref{lem:sumDiffRoot} folgt, dass $\beta + \alpha'$ eine positive Wurzel ist.
  Dies widerspricht aber der vorausgesetzten Maximalität von $\beta$, die anfängliche Annahme $\Delta_2$ sei nicht leer muss also verworfen werden.
  Also sind alle Linearfaktoren $k_\alpha$ strikt positiv.

  Zudem zeigt der vorangehende Beweis, dass $(\alpha', \beta) \geq 0$ gilt für alle $\alpha' \in \Delta$ und, dass zumindest ein $\alpha' \in \Delta$ existiert mit $(\alpha', \beta) > 0$.
    Denn $\Delta$ ist nach \hyperref[it:B2]{(B2)} eine Vektorraumbasis für $E$ und falls $(\alpha', \beta) = 0$ gilt für alle $\alpha' \in \Delta$, folgt sogleich $(v, \beta) = 0$ für alle $v \in E$ und damit $\beta = 0$, da Skalarprodukte nicht ausgeartet sind.

  Nun wollen wir die Eindeutigkeit der maximalen Wurzel zeigen.
  Sei dazu $\beta'$ eine weitere Wurzel.
  Nach \hyperref[it:B2]{(B2)} und den vorangehenden Ausführungen existiert wieder eine Darstellung der Form $\beta' = \sum_{\alpha \in \Delta} k_\alpha' \alpha$ mit $k_\alpha' > 0$ für alle $\alpha \in \Delta$.
  Ebenso existiert mindestens ein $\alpha' \in \Delta$ mit $(\alpha', \beta') > 0$.
  Damit folgt nun
  \begin{displaymath}
    (\beta, \beta') = \sum_{\alpha \in \Delta} k_\alpha (\alpha, \beta') \geq k_{\alpha'} (\alpha', \beta') > 0.
  \end{displaymath}
  Mit Lemma \ref{lem:sumDiffRoot} ergibt sich nun, dass auch $\beta - \beta'$ eine Wurzel ist außer $\beta$ und $\beta'$ sind proportional, also $\beta = \beta'$, da beide Wurzeln voraussetzungsgemäß positiv sind.
  Angenommen $\beta \neq \beta'$ und $\beta - \beta' \succeq 0$
  Dann wäre auch $\beta \succeq \beta'$ im Widerspruch zur Maximalität von $\beta$.
  Umgekehrt zeigt man, das auch die Annahme $\beta' \succeq \beta$ verworfen werden muss.
  Also gilt $\beta = \beta'$ und die maximale Wurzel ist eindeutig.
\end{proof}

\begin{lem}
  \label{lem:irreducibleGroupOp}
  Sei $\Phi$ ein irreduzibel.
  Dann ist auch die Gruppenoperation von $\W$ irreduzibel über $E$, es existieren also außer $\{0\}$ und $E$ keine weiteren $\W$\hyp{}invarianten Unterrräume von $E$.
  Insbesondere erzeugt die $\W$\hyp{}Bahn einer beliebigen Wurzel $\alpha$ den gesamten Vektorraum $E$.
\end{lem}

\begin{proof}
  Sei $E'$ ein nichttrivialer $\W$\hyp{}invarianter Untervektorraum von $E$
  Dann ist auch das orthogonale Komplement $(E')^\perp$ ein $\W$\hyp{}invarianter Unterraum.
  Ist nämlich $\alpha \in E'$ und $\beta \in (E')^\perp$, so gilt für $\sigma \in \W$ die Gleichung
  $\langle \alpha, \sigma(\beta) \rangle = \langle \sigma^{-1}(\alpha), \beta \rangle = 0$, da $\sigma^{-1}(\alpha) \in E'$ gilt.
  Es folgt also auch $\sigma(\beta) \in (E')^\perp$ und damit die $\W$\hyp{}Invarianz von $(E')^\perp$.

  Zudem gilt die für orhtogonale Komplemente gültige Identität $E = E' \oplus (E')^\perp$.

  Dann ist aber auch $\Phi = (\Phi \cap E') \cup (\Phi \cap (E')^\perp$ eine Zerlegung von $\Phi$ in zueinander orthogonale Teilmengen.
  Da aber $\Phi$ als irreduzibles Wurzelsystem vorausgesetzt wurde mus eine der beteiligten Teilmengen leer sein.
  Angenommen $\Phi \subseteq E'$. 
  Da nach \hyperref[it:W1]{(W1)} Wurzelsysteme den Vektorraum erzeugen, gilt $E = E'$. 
  Ebenso folgt im Falle $\Phi \subseteq (E')^\perp$ die Aussage $E = (E')^\perp$.
  Damit haben wir den ersten Teil der Behauptung bewiesen.

  Es ist der Spann einer $\W$\hyp{}Bahn einer Wurzel $\alpha$ ein nicht trivialer $\W$\hyp{}invarianter Untervektorraum von $E$.
  Mit dem vorangehenden Beweis musss $\spn(\W\alpha) = E$ gelten.
  Damit ist die Behauptung vollständig bewiesen.
\end{proof}

\begin{bem}
  Der Beweis zeigt bereits auf, wie sich reduzible Wurzelsysteme aus irreduziblen Wurzelsystemen zusammensetzen.
  Es korrespondiert nämlich zu jeder disjunkten Zerlegung des Wurzelsystems eine orthogonale Summe $\W$\hyp{}invarianter Untervektorräume.
  Eine Weiterführung dieses Gedankens zur Klassifikation von Wurzelsystemen findet sich in \cite[S.57ff.]{humphreys1972introduction}.
\end{bem}

\begin{lem}
  \label{lem:rootLength}
  Sei $\Phi$ irreduzibel.
  Es bezeichne $||\alpha|| := \sqrt{(\alpha,\alpha)}$ die vom Skalarprodukt auf $E$ induzierte Norm.
  Dann nimmt die Norm auf $\Phi$ höchstens zwei unterschiedliche Werte an.
  Zudem liegen alle Wurzeln mit gleicher Norm in einer gemeinsamen $\W$\hyp{}Bahn.
\end{lem}

\begin{proof}
  Seien $\alpha, \beta \in \Phi$.
  Dann können nicht alle Elemente aus der Bahn $\W\alpha$ orthogonal zu $\beta$ sein, da nach Lemma \ref{lem:irreducibleGroupOp} die Bahn $\W\alpha$ den Vektorraum $E$ erzeugt.
  Aus \cite[S.45]{humphreys1972introduction} ist bekannt, dass  
  \begin{displaymath}
    \frac{(\beta,\beta)}{(\alpha,\alpha)} \in \left\{1,2,3,\frac{1}{2},\frac{1}{3}\right\} \tag{$\ast$}
  \end{displaymath}
  vorausgesetzt $(\alpha, \beta) \neq 0$.
  Daraus folgt bereits der erste Teil der Behauptung.
  Denn, angenommen es existieren drei unterschiedliche Wurzellängen, dann müsste auch das Verhältnis $\tfrac{3}{2}$ in ($\ast$) auftauchen, was jedoch nicht der Fall ist.

  Seien nun $\alpha, \beta$ Wurzeln gleicher Norm.
  Da nach Lemma \ref{lem:irreducibleGroupOp} die Bahn $\W\alpha$ den Vektorraum $E$ erzeugt, können wir $(\alpha, \beta) \neq 0$ annehmen.
  Ansonsten ersetzen wir $\alpha$ durch ein $\W$\hyp{}Bahnelement $\sigma(\alpha)$ mit $(\sigma(\alpha), \beta) \neq 0$.
  Im Falle $\alpha = \beta$ gibt es nichts zu zeigen, daher nehmen wir an es gelte $\alpha \neq \beta$.
  Nun folgt unter Berücksichtigung von \cite[S.45]{humphreys1972introduction}, dass $\langle \alpha, \beta \rangle = \langle \beta, \alpha \rangle = \pm 1$ gelten muss.
  Aufgrund im vorigen Ausdruck für $\alpha$ und $\beta$ geltenden Symmetrie ist der im Allgemeinen nur im ersten Argument lineare Term $\langle \alpha, \beta \rangle$ nun sogar bilinear.
  Wir können damit falls nötig $\beta$ auch durch $-\beta = \sigma_\beta(\beta)$ ersetzen, da diese Elemente in derselben $\W$\hyp{}Bahn liegen.
  Daher lässt sich annehmen, dass $\langle \alpha, \beta \rangle = 1$ gilt.
  Damit folgt
  \begin{displaymath}
    \sigma_\alpha \sigma_\beta \sigma_\alpha(\beta) 
    = \sigma_\alpha \sigma_\beta(\beta - \langle \beta, \alpha \rangle \alpha)
    = \sigma_\alpha \sigma_\beta(\beta - \alpha)
    = \sigma_\alpha (-\beta - \alpha + \beta)
    = \alpha
  \end{displaymath}
  und es liegen $\alpha$ und $\beta$ in der selben $\W$\hyp{}Bahn.
\end{proof} 

\begin{bem}
  Ist das Wurzelsystem $\Phi$ irreduzibel und mit Wurzeln unterschiedlicher Norm, so ist es auf Grundlage von Lemma \ref{lem:rootLength} möglich von \emph{langen} und \emph{kurzen} Wurzeln zu sprechen.
  Haben alle Wurzeln aus $\Phi$ dieselbe Norm, so ist es üblich alle Wurzeln als lang zu bezeichnen.
\end{bem}

\begin{lem}
  Sei $\Phi$ irreduzibel und enthalte Wurzeln unterschiedlicher Norm.
  Dann ist die im Sinne von Lemma \ref{lem:maximalRoot} maximale Wurzel $\beta$ eine lange Wurzel.
\end{lem}

\begin{proof}
  Sei $\alpha \in \Phi$ gegeben.
  Da die Normabbildung eingeschränkt auf $\Phi$ nur zwei Werte annimmt, reicht es aus $(\beta, \beta) \geq (\alpha, \alpha)$ zu zeigen.
  Dazu ersetzen wir $\alpha$ und $\beta$ durch die nach \ref{lem:fundamentalDomain} eindeutig Bestimmten Wurzeln aus $\W\alpha$ beziehungsweise $\W\beta$, die im Abschluss der Fundamentalkammer bezüglich $\Delta$ liegen. 
  Dies ist ohne Einschränkungen möglich, da Lemma \ref{lem:rootLength} garantiert, dass alle Bahnelemente dieselbe Norm besitzen.
  Nach Lemma \ref{lem:maximalRoot} gilt aufgrund der Maximalität von $\beta$ nun $\beta - \alpha \succeq 0$.
  Weiter ist $(\gamma, \beta - \alpha) \geq 0$ für alle $\gamma \in \overline{\mathfrak{C}(\Delta)}$, also insbesondere auch für $\alpha$ und $\beta$, was zu 
  \begin{displaymath}
    (\beta, \beta) - (\beta, \alpha) \geq 0 \quad\text{und}\quad 
    (\beta, \alpha) - (\alpha, \alpha) 
    = (\alpha, \beta) - (\alpha, \alpha) \geq 0
  \end{displaymath}
  führt.
  Beide ungleichungen zusammen ergeben folglich
  \begin{displaymath}
    (\beta, \beta) \geq (\beta, \alpha) \geq (\alpha, \alpha),
  \end{displaymath}
  was die Behauptung beweist.
\end{proof}
